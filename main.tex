%!TEX encoding = UTF-8 Unicode
\documentclass[ucs,9pt,english,ngerman,xcolor=table]{beamer}

%\usepackage[utf8x]{inputenc}    % to make utf-8 input possible
%\usepackage{babel}     % hyphenation etc., alternatively use 'german' as parameter

% //////////////////// Pakete laden ////////////////////
\usepackage{amsmath}			% MUSS vor fontspec geladen werden
\usepackage{mathtools}			% modifiziert amsmath
\usepackage{amssymb}			% mathematische symbole, für \ceckmarks
\usepackage{amsthm}				% für proof
\usepackage{mathrsfs}			% für \mathscr
\usepackage{latexsym}

\usepackage{fontspec} 			% funktioniert nur mit den neueren Compilern z.B. XeLaTeX
%\usepackage[T1]{fontenc}
%\usepackage[utf8]{inputenc}		% funktioniert mit PDFLaTeX
\usepackage{microtype}			% für bessere Worttrennung
\usepackage[ngerman]{babel} 	% Spracheinstellung
\usepackage{lmodern}			% verändert verwendete Schriftart, damit sie weniger pixelig ist

\usepackage{verbatim}
\usepackage{listings}			% Für Quellcode
\usepackage{tabularx}

\usepackage{graphicx}
\usepackage{multirow}			% für multirow in tabulars

\PassOptionsToPackage{cmyk,table}{xcolor} % weil xcolor von anderem paket bereits geladen wird
%\usepackage[cmyk,table]{xcolor} % um Farben zu benutzen, kann mehr als das Paket color
\PassOptionsToPackage{colorlinks,linktocpage,linkcolor=blue}{hyperref}
% \usepackage[					% Verlinkungen
% 	colorlinks,					% farbige Schrift, statt farbiger Rahmen
% 	linktocpage,				% verlinkt im Abb.Verzeichnis Seitenzahl statt Bildunterschrift
% 	linkcolor=blue				% setzt Farbe der Links auf blau
% 	]{hyperref}					% nur für digitale Anwendungen, url = "http://www.example.com"
\usepackage{url}				% für Webadressen wie e-mail usw.: "\url{http://www.example.com}"
\renewcommand\UrlFont{\footnotesize\color{FUblue}\rmfamily}

\usepackage{enumerate}			% für versch. Aufzählungezeichen wie z.B. a)
\usepackage{float}
\usepackage{xspace}				% folgt ein Leerzeichen nach einem \Befehl, wird es nicht verschluckt.
\usepackage{multicol}			% für mehrspaltiges Layout

\usepackage{fp}
\usepackage{tikz}
\usetikzlibrary{tikzmark}			% für \tikzmark{toRemember}
\usetikzlibrary{positioning}	% verbesserte Positionierung der Knoten
\usetikzlibrary{automata}		% für Automaten (GTI)
\usetikzlibrary{arrows}
\usetikzlibrary{shapes}
\usetikzlibrary{decorations.pathmorphing}
\usetikzlibrary{decorations.pathreplacing}
\usetikzlibrary{decorations.shapes}
\usetikzlibrary{decorations.text}

% //////////////////// eigene Farben ////////////////////
\let\definecolor=\xdefinecolor
\definecolor{FUgreen}{RGB}{153,204,0}
\definecolor{FUblue}{RGB}{0,51,102}

\definecolor{middlegray}{rgb}{0.5,0.5,0.5}
\definecolor{lightgray}{rgb}{0.8,0.8,0.8}
\definecolor{orange}{rgb}{0.8,0.3,0.3}
\definecolor{azur}{rgb}{0,0.7,1}
\definecolor{yac}{rgb}{0.6,0.6,0.1}
\definecolor{Pink}{rgb}{1,0,0.6}

\definecolor{bgcolour}{rgb}{0.97,0.97,0.97}
\definecolor{codegreen}{rgb}{0,0.6,0}
\definecolor{codegray}{rgb}{0.35,0.35,0.35}
\definecolor{codepurple}{rgb}{0.58,0,0.82}
\definecolor{codeblue}{rgb}{0.4,0.5,1}

% //////////////////// Syntaxhighlighting ////////////////////
\lstloadlanguages{Python, Haskell, [LaTeX]TeX, Java}
\lstset{
	basicstyle=\small\fontfamily{fvm}\selecfont,	% \scriptsize the size of the fonts that are used for the code
	backgroundcolor = \color{bgcolour},	% legt Farbe der Box fest
	breakatwhitespace=false,	% sets if automatic breaks should only happen at whitespace
	%breaklines=true,			% sets automatic line breaking
	captionpos=t,				% sets the caption-position to bottom, t for top
	commentstyle=\color{codeblue}\ttfamily,% comment style
	frame=single,				% adds a frame around the code
	keepspaces=true,			% keeps spaces in text, useful for keeping indentation
	% of code (possibly needs columns=flexible)
	keywordstyle=\bfseries\color{blue},% keyword style
	numbers=left,				% where to put the line-numbers;
	% possible values are (none, left, right)
	numberstyle=\tiny\color{codegreen},	% the style that is used for the line-numbers
	numbersep=5pt,			% how far the line-numbers are from the code
	stepnumber=1,				% nummeriert nur jede i-te Zeile
	showspaces=false,			% show spaces everywhere adding particular underscores;
	% it overrides 'showstringspaces'
	%showstringspaces=false,	% underline spaces within strings only
	showtabs=false,			% show tabs within strings adding particular underscores
	flexiblecolumns=false,
	%tabsize=1,				% the step between two line-numbers. If 1: each line will be numbered
	stringstyle=\color{orange}\ttfamily,	% string literal style
	numberblanklines=false,				% leere Zeilen werden nicht mitnummeriert
	xleftmargin=1.2em,					% Abstand zum linken Layoutrand
	xrightmargin=0.4em,					% Abstand zum rechten Layoutrand
	aboveskip=2ex, 
}

\lstdefinestyle{py}{
	language=Python,
}
\lstdefinestyle{hs}{
	language=Haskell,
}
\lstdefinestyle{tex}{
	language=[LaTeX]TeX,
	escapeinside={\%*}{*)},     % if you want to add LaTeX within your code
	texcsstyle=*\bfseries\color{blue},% hervorhebung der tex-Schlüsselwörter
	morekeywords={*,$,\{,\},\[,\],lstinputlisting,includegraphics,listoffigures,lstlistoflistings,subsection,subsubsection,textcolor,tableofcontents,colorbox,fcolorbox,definecolor,cellcolor,url,linktocpage,subtitle,subject,maketitle,usetikzlibrary,node,path,addbibresource,printbibliography},% if you want to add more keywords to the set
}
\lstdefinestyle{java}{
	language=Java,
	extendedchars=true,		% lets you use non-ASCII characters;
	% for 8-bits encodings only, does not work with UTF-8
}

% //////////////////// eigene Kommandos ////////////////////
\newcommand{\Quellcode}[3]{\lstinputlisting[style=#2, caption={#3}]{#1.#2}}% 1. path/filename, 2. type, 3. beschr.
\newcommand\zz{\ensuremath{\raisebox{+0.25ex}{Z}% zu zeigen
		\kern-0.4em\raisebox{-0.25ex}{Z}%
		\;\xspace}%
}
\newenvironment{Magic}[1][Pink]% hat 1 optionales Arg., Standardwert: Pink
{\begin{center}\begingroup\color{#1}\huge}%
{\endgroup\end{center}}
\input{src/fub-template}  % THIS is the line that includes the FU template!

\usepackage{arev,t1enc} % looks nicer than the standard sans-serif font
%if you experience problems, comment out the line above and change
%the documentclass option "9pt" to "10pt"

% image to be shown on the title page (without file extension, should be pdf or png)
\titleimage{img/latex}


% (optional, use only with long paper titles)
\title[\LaTeX{} für Seminararbeiten]{Grundlagen \LaTeX{} für Seminararbeiten \\ im Fachbereich Wirtschaftswissenschaft}
%\subtitle{Mentoring WiSe 2018}
\author{Sebastian Engelmann}
\institute[Freie Universität Berlin] % (optional, but mostly needed)
{Freie Universität Berlin}
\date{8. Mai 2018}
%\subject{Computer Science}

% you can redefine the text shown in the footline. use a combination of
% \insertshortauthor, \insertshortinstitute, \insertshorttitle, \insertshortdate, ...
% \renewcommand{\footlinetext}{\insertshortinstitute, \insertshorttitle}

\AtBeginSection[]
{
	\begin{frame}<beamer>{Inhalt}
	\begin{multicols}{2}%  erstellt 2 Spalten
		\tableofcontents[currentsection,currentsubsection]
	\end{multicols}
\end{frame}
}
% Delete this, if you do not want the table of contents to pop up at
% the beginning of each subsection:
\AtBeginSubsection[]
{
\begin{frame}<beamer>{Inhalt}
\begin{multicols}{2}%  erstellt 2 Spalten
	\tableofcontents[currentsection,currentsubsection]
\end{multicols}
\end{frame}
}

% //////////////////// BEGIN DOKUMENT ///////////////////////
\begin{document}

\begin{frame}[plain]
\titlepage
\end{frame}

\begin{frame}{Inhalt}%
\begin{multicols}{2}%  erstellt 2 Spalten
\tableofcontents
\end{multicols}
% You might wish to add the option [pausesections]
\end{frame}

\section{Was ist \LaTeX?}

\begin{frame}{Was ist \LaTeX?}
\begin{itemize}
    \item \LaTeX{} ist eine Sammlung von \textbf{Makros}, die die Nutzung von \TeX{} ermöglicht
    \newline
    \item \TeX{} wiederum ist ein \textbf{Drucksatzsystem}, mit dem sich Texte formatieren lassen
    \newline
    \item Text und Formatierungselemente werden \textbf{direkt in den Quelltext} geschrieben
    \newline
    \item \textbf{Vorteile:} Einfachheit in der Darstellung komplexer Strukturen, mathematischer Formeln, Grafiken und ähnlichem
\end{itemize}
    
\end{frame}

\begin{frame}{Was ist \LaTeX?: Installation}
\begin{itemize}
\item \textbf{Windows:}
\begin{enumerate}
\item \textbf{MikTex Installer} installieren: \url{http://miktex.org/download}
\item einzelne \textbf{Pakete} installieren: \textit{"MiKTeX" > "Maintenance (Admin)" > "Package Manager (Admin)"}
\item \LaTeX{} \textbf{Editor} einrichten, z.B. Texmaker
\end{enumerate}
\item \textbf{Linux:}
\begin{enumerate}
\item \$ sudo apt-get install texlive-full
\item \$ sudo apt-get install texlive-xetex
\item \$ sudo apt-get install texlive texlive-doc-de texlive-latex-extra texlive-lang-german
\item \$ sudo apt-get install latex-xcolor pgf tex-common texlive texlive-base texlive-base-bin texlive-common texlive-fonts-extra texlive-fonts-recommended texlive-lang-german texlive-latex-base texlive-latex-extra texlive-latex-recommended
\item \$ sudo apt-get install texlive-pictures dot2tex sketch libqtexengine1 texlive-humanities texlive-pstricks
\item \$ wget \url{http://mirror.ctan.org/graphics/pgf/contrib/tikz-qtree.zip}
\begin{itemize}
\item \textbf{im Ordner entpacken}
\end{itemize}

\item \$ sudo mv tikz-qtree/ /usr/share/texmf/tex/latex/
\item \$ sudo texhash
\end{enumerate}
\end{itemize}
\end{frame}

\section{Funktionsweise}

\begin{frame}{Funktionsweise}
Für einfache Texte gelten folgende Schritte:
\begin{enumerate}
	\item Texte im Editor schreiben, bspw. mit \textit{Kile}: \texttt{kile meinText.tex}
	\item Dokument übersetzen, bspw. mit pdf\LaTeX{}: \texttt{pdflatex meinText.tex}
	\newline
\end{enumerate}
Es werden dabei mindestens folgende Dateien erzeugt:
\begin{tabular}[pos]{lp{8.5cm}}
	\texttt{meinText.log} & Enthält alle Statusmeldungen des Übersetzungsvorgangs. \\
	\texttt{meinText.aux} & Enthält unter anderem die Einträge für Querverweise. \\
	\texttt{meinText.pdf} & Das erzeugte PDF-Dokument (nach dem ersten Durchlauf ohne Inhaltsverzeichnis).
\end{tabular}
\begin{enumerate}
	\item[3.] Erneutes Übersetzen des Dokuments: \texttt{pdflatex meinText.tex}
	\item[4.] PDF-Dokument ansehen, bspw. mit dem PDF-Viewer \textit{okular}: \texttt{okular meinText.pdf}
\end{enumerate}
\end{frame}

\begin{frame}{Funktionsweise}
\begin{itemize}
	\item Je nach Komplexität besteht die Notwendigkeit mehrerer Durchläufe des \TeX{}-Compilers
	\newline
	\item Beim ersten Durchlauf erzeugt der \TeX{}-Compiler aus allen Überschriften das Inhaltsverzeichnis (Die Informationen werden in einer Hilfdatei gespeichert -- Dataiendung \texttt{.aux})
	\newline
	\item Da das Inhaltsverzeichnis am Anfang des Dokuments steht, liegt es beim ersten Start noch nicht vor
	\newline
	\item Erst beim zweiten Durchlauf wird das Inhaltsverzeichnis eingebunden
	\newline
	\item Dies kann aber dazu führen, dass alle nachfolgenden Seiten nach hinten verschoben werden, wenn das Inhaltsverzeichnis mehrere Seiten umfasst und so eventuell Verweise auf Seitenzahlen nicht mehr stimmen
	\newline
	\item Somit muss ein dritter Durchlauf gestartet werden
\end{itemize}
\end{frame}

\section{Aufbau eines \LaTeX{}-Dokuments}

% ===== Struktur =====
\subsection{Struktur}

\begin{frame}[fragile]{Struktur}
	Die Minimalstruktur eines Quelltextes besteht aus zwei wesentlichen Teilen: Der \textbf{Präambel} und dem \textbf{Textkörper}. Alles was zwischen \texttt{\textbackslash documentclass} und \texttt{\textbackslash begin\{document\}} steht, zählt formal zur Präambel und alles danach zum Textkörper. 	
\begin{lstlisting}[style=tex]
\documentclass[Optionen]{Name}[Version] 
...
\begin{document}
...
\end{document}
\end{lstlisting}
Die Präambel enthält prinzipiell all das, was an allgemeinen Definitionen und zusätzlichen festlegungen für das gesamte Dokument vorgenommen wird.	
\end{frame}

% ===== Beispiele =====
\subsection{Beispiele}

\begin{frame}[fragile]{Ein pdf\LaTeX{}-Beispiel}
\begin{lstlisting}[style=tex]
%% Ein pdflatex-Beispiel
\documentclass[a5paper, ngerman, 11pt]{article}
\usepackage[T1]{fontenc}
\usepackage{lmodern}
\usepackage[utf8]{inputenc}
\usepackage{microtype}
\usepackage{geometry}
\usepackage{babel}
\usepackage{eurosym}
\usepackage{blindtext}
\title{Textsatz mit \LaTeX}
\author{Johannes Gutenberg}
\date{\today}
\begin{document}
\maketitle
\section{Einführung}
Erste Versuche mit dem Setzen eines \LaTeX-Dokuments für 0,--\,\euro.
\blindtext[2]
\end{document}
\end{lstlisting}
\end{frame}

\begin{frame}[fragile]{Erläuterung des pdf\LaTeX{}-Beispiels}
\begin{lstlisting}[style=tex, basicstyle=\tiny]
\documentclass[a5paper, ngerman, 11pt]{article}
   Dokumentenklasse article mit den globalen Optionen a5paper für das Papierformat, 
   ngerman für die neue deutsche Rechtschreibung und 11pt für die Grundschriftgröße.
\usepackage[T1]{fontenc}
   Lade das Paket fontenc für die Schriftkodierung (fontencoding) mit der Option T1, damit 
   zum einen Vektorschriften eingebunden werden und zum anderen eine Trennung an 
   deutschen Umlauten ermöglicht wird.
\usepackage[utf8]{inputenc}
   Lade das Paket inputenc für die Eingabekodierung (inputencoding) mit der Option utf8, 
   damit Umlaute und das ß direkt über die Tastatur eingegeben werden können.
\usepackage{babel}
   Lade Sprachpaket babel, welches die globale Option ngerman der Dokumentenklasse 
   auswertet.
\title{Textsatz mit \LaTeX}
   Definiere den Titel.
\author{Johannes Gutenberg}
   Definiere den Autor
\date{\today}
   Definiere das Datum. Es wird hier das aktuelle Datum verwendet.
\begin{document}
   Beginn des Textkörpers.
\maketitle
   Formatierten Titel ausgeben.
\section{Einführung}
   Abschnittsüberschrift.
\end{document}
   Ende des Dokuments.
\end{lstlisting}
\end{frame}

\begin{frame}[fragile]{Ein \XeLaTeX{}-Beispiel}
\begin{lstlisting}[style=tex]
%% Ein xelatex-Beispiel
\documentclass[a5paper, ngerman, 11pt]{article}
\usepackage{fontspec}
\fontspec[Mapping=tex-text]{Linux Libertine O}
\usepackage{microtype}
\usepackage{polyglossia}
\setmainlanguage[spelling=new]{german}
\usepackage{geometry}
\usepackage{dtk-logos}
\usepackage{blindtext}
\title{Textsatz mit \XeLaTeX}
\author{Johannes Gutenberg}
\date{\today}
\begin{document}
\maketitle
\section{Einführung}
Erste Versuche mit dem Setzen eines \XeLaTeX-Dokuments für 0,--.
\blindtext[2]
\end{document}
\end{lstlisting}
\end{frame}

\begin{frame}[fragile]{Ein \LuaLaTeX{}-Beispiel}
\begin{lstlisting}[style=tex]
%% Ein lualatex-Beispiel
\documentclass[a5paper, ngerman, 11pt]{article}
\usepackage{fontspec}
\setmainfont[Ligatures=TeX]{Linux Libertine O}
\usepackage{microtype}
\usepackage{babel}
\usepackage{geometry}
\usepackage{dtk-logos}
\usepackage{blindtext}
\title{Textsatz mit \LuaLaTeX}
\author{Johannes Gutenberg}
\date{\today}
\begin{document}
\maketitle
\section{Einführung}
Erste Versuche mit dem Setzen eines \LuaLaTeX-Dokuments für 0,--.
\blindtext[2]
\end{document}
\end{lstlisting}
\end{frame}

% ===== Dokumentenklassen =====
\subsection{Dokumententypen}

\begin{frame}[fragile]{Dokumentenklassen}
In jedem \LaTeX{}-Dokument muss die \textbf{Dokumentenklasse} angegeben werden (erster Befehl des Dokuments). Jede Dokumentenklasse hat \textbf{verschiedene Eigenschaften}. Die Dokumentenklasse (in geschweiften Klammern) wird über den Befehl \texttt{\textbackslash documentclass} ausgewählt. Der ausgewählten Klasse können Optionen (in eckigen Klammern) mitgegeben werden, die von der Dokumentenklasse selbst als auch von den verwendeten Paketen ausgewertet werden können.
%\begin{varwidth}{0.45\linewidth}
\begin{lstlisting}[style=tex]
\documentclass[Optionen]{Dokumentenklasse}
\end{lstlisting}
\end{frame}

% ////////////////// Typen Übersicht //////////////
\begin{frame}{Standardklassen}
\begin{itemize}
	\item \textbf{book:} Drucklayout standardmä{\ss}ig zweiseitig; z.B.: automatische Kopfzeile mit Seitenzahl
	\begin{itemize}
		\item Titel auf eigener Seite
		\item Seitenaufzählungen mit römischen und arabischen Ziffern
		\item Ebenen \textbackslash\texttt{part}, \textbackslash\texttt{chapter}, \textbackslash\texttt{section}, \textbackslash\texttt{subsection}, \textbackslash\texttt{subsubsection} und \textbackslash\texttt{paragraph}
	\end{itemize}
\item \textbf{article:} geeignet für kurze technische Artikel
	\begin{itemize}
		\item Titel auf eigener Seite
		\item Seitenzählung mit arabischen Ziffern
		\item Ebenen \textbackslash\texttt{part}, \textbackslash\texttt{section}, \textbackslash\texttt{subsection}, \textbackslash\texttt{paragraph}
		\item Fortlaufende Nummerierungen der Abbildungen, Tabellen, Fu{\ss}noten und Gleichungen
	\end{itemize}
\item \textbf{report:} geeignet für längere technische Artikel
	\begin{itemize}
		\item wie \texttt{article}, jedoch mit \textbackslash\texttt{chapter}
	\end{itemize}
\item \textbf{letter:} geeignet zum Schreiben von Briefen
	\begin{itemize}
		\item Keine Ebenen
		\item Elemente, die ein Brief aufweist, wie Absender, Anschrift, usw.
	\end{itemize}
\end{itemize}
\end{frame}

\begin{frame}{Koma-Script}
	Die Koma-Script-Reihe ist eine Sammlung von Klassen und Paketen, die besonders die typografischen Gepflogenheiten eines europäischen Layouts berücksichtigen (\url{https://komascript.de}). Sie besteht aus folgenden Dokumentenklassen: \\
	%\vspace{0.5cm}
	\begin{itemize}
		\item \textbf{scrbook}
		\item \textbf{scrartcl}
		\item \textbf{scrreprt}
		\item \textbf{scrlettr2}
	\end{itemize}

\end{frame}

% ///////////// Pakete ///////////////
\subsection{Pakete}
\begin{frame}[fragile]{Pakete}
\TeX{} und \LaTeX{} besitzen einen Pool an Makros (Befehle). Wenn man darüber hinaus Modifizierungen vornehmen möchte, muss man zusätzliche Pakete einbinden, die weitere Makros zur Verfügung stellen. Sie werden in die Präambel eingefügt. %\pause

\begin{lstlisting}[style=tex]
\usepackage{fontspec}
% Schriftpaket, funktioniert nur mit den neueren Compilern z.B. XeLaTeX
\usepackage{microtype}
% verbessert die Worttrennung
\usepackage[ngerman]{babel}
% Spracheinstellung: richtige Silbentrennung.
\usepackage{lmodern}
% verändert verwendete Schriftart, damit sie weniger pixelig ist
\end{lstlisting}
\end{frame}

% ===== Titlepage =====
\subsection{Titelseite}
\begin{frame}[fragile]{Titelseite}
Die Angaben für die Titelseite, wie bspw. Titel, Autor und Datum lassen sich sehr einfach erstellen und werden dann automatisch vorformatiert.

\begin{lstlisting}[style=tex]
\begin{titlepage}
\title{Grundlagen \LaTeX{} für Seminararbeiten im Fachbereich Wirtschaftswissenschaft}
\subtitle{Mentoring SoSe 2018}
\subject{Fachbereich Wirtschaftswissenschaft}
\author{Sebastian Engelmann}
\date{8. Mai 2018}
\end{titlepage}

\maketitle % erstellt die Titelseite
\end{lstlisting}

% \textbf{Anwendung:}
% \vspace{-2ex}
% \begin{lstlisting}[style=tex]
% \begin{titlepage}
%   ...
% \end{titlepage}

% \begin{document}
%    \maketitle
% \end{document}
% \end{lstlisting}
% Wendet man die Zeilen 1 -- 7 vor Beginn des Dokuments an, wird eine extra Cover-Seite erstellt.

% \begin{lstlisting}[style=tex]
% \begin{document}
%   \begin{titlepage}
%     ...
%   \end{titlepage}

%   \maketitle
% \end{document}
% \end{lstlisting}
% Wendet man es innerhalb des Dokuments an, wird nachfolgender Inhalt auf die selbe Seite gesetzt.
\end{frame}

% ===== Sonderzeichen =====
\subsection{Sonderzeichen}

\begin{frame}{Sonderzeichen in \LaTeX}
Nicht alle Zeichen der Tastatur stehen für die Eingabe zur Verfügung, da einige Sonderzeichen als \TeX{}-Steuerzeichen verwendet werden. Die nachfolgende Tabelle fasst die reservierten Zeichen zusammen.
\begin{table}
\caption{Eingabe von Sonderzeichen in \LaTeX}
	\begin{tabular}{cll}
		\hline
		Zeichen & Eingabe & Bedeutung \\
		\hline
		\{ & \textbackslash\{ & Beginn einer Gruppe \\
		\} & \textbackslash\} & Ende einer Gruppe \\
		\# & \textbackslash\# & Parameter \\
		\& & \textbackslash\& & Trenner bei Matrizen und Tabellen \\
		\_ & \textbackslash\_ & Indizes im mathematischen Modus \\
		\% & \textbackslash\% & Leitet Kommentare ein \\
		\$ & \textbackslash\$ & Mathematischer Zeilenmodus \\
		\textbackslash & \textbackslash\texttt{textbackslash} & Makrobeginn \\
		\textasciitilde & \textbackslash\texttt{textasciitilde} & Leerzeichen ohne Umbruch \\
	    \textasciicircum & \textbackslash\texttt{textasciicircum} & Exponent im mathematischen Modus \\
		\hline
	\end{tabular}
\end{table}
\end{frame}
\input{slides/makros_und_umgebungen}
\section{Textstrukturierung}

% ===== Gliederung =====
\subsection{Gliederung \& Inhaltsverzeichnis}
\begin{frame}[fragile]{Gliederung \& Inhaltsverzeichnis}
%\begin{varwidth}{0.45\linewidth}
%Eine Überschrift wird mit folgendem Befehl erstellt:
Mit dem Makro \texttt{\textbackslash section} wird eine Überschrift in der ersten Gliederungsebene erzeugt:
\begin{lstlisting}[style=tex]
\section{head}
\end{lstlisting}
%\pause
Für eine Überschrift in der zweiten Gliederungsebene benutzen wir das Makro \texttt{\textbackslash subsection}:
\begin{lstlisting}[style=tex]
\subsection{subhead}
\end{lstlisting}
%\pause
In der dritten Gliederungsebene benutzen wir das Makro \texttt{\textbackslash subsubsection} für eine Überschrift:
\begin{lstlisting}[style=tex]
\subsubsection{subsubhead}
\end{lstlisting}
\end{frame}

% ===== Inhaltsverzeichnis =====
\begin{frame}[fragile]{Gliederung \& Inhaltsverzeichnis}
Das Inhaltsverzeichnis wird automatisch mit dem Makro
\begin{lstlisting}[style=tex]
\tableofcontents
\end{lstlisting}
erstellt. Dabei werden alle Überschriften übernommen.
%\pause
\newline
\newline
Möchte man nicht, dass eine Überschrift im Verzeichnis auftaucht, benutzt man das Sternchen:
\begin{lstlisting}[style=tex]
\section*{diese Überschrift taucht nicht auf}
\end{lstlisting}
\end{frame}

% ===== Absätze, Zeilenumbrüche =====
\subsection{Absätze \& Zeilenumbrüche}
\begin{frame}[fragile]{Absätze \& Zeilenumbrüche}

Ein Zeilenumbruch kann entweder durch eine \textbf{Leerzeile} im Quelltext oder durch das Makro \textbf{\textbackslash\texttt{par}}  erreicht werden.

\begin{itemize}
	\item Gliederung des Inhalts in Gedankengänge
	\item 1. Zeile ist eingerückt
	\item Abstand zwischen zwei Absätzen
\end{itemize}

%\begin{multicols}{2}
%\begin{lstlisting}[style=tex]
%\par
%\end{lstlisting}
%\begin{lstlisting}[style=tex]
%\newline
%\end{lstlisting}
%\begin{lstlisting}[style=tex]
%\linebreak
%\end{lstlisting}
%\begin{lstlisting}[style=tex]
%\\
%\end{lstlisting}
%\end{multicols}%\pause

Möchte man nicht, dass die erste Zeile eines Absatzes eingerückt wird, kann man folgende Zeile in die Präambel des Dokuments vornehmen:
\begin{lstlisting}[style=tex]
\parindent 0pt
\end{lstlisting}

Einen manuellen Zeilenumbruch ohne Abstand kann man mit folgenden Makros erzeugen:
\begin{lstlisting}[style=tex]
\newline              % Zeilenumbruch im selben Absatz
\linebreak
\\
\end{lstlisting}

\end{frame}

% ===== Auflistung, Aufzählung =====
\subsection{Auflistung \& Aufzählung}
\begin{frame}[fragile]{Auflistung \& Aufzählung}
Für unnummerierte Auflistungen kann man die Umgebung \textbf{itemize} benutzen. Sie kann bis zu vier Ebenen tief geschachtelt werden.
\begin{lstlisting}[style=tex]
\begin{itemize}
\item Stichpunkt 1
\item Stichpunkt 2
...
\end{itemize}
\end{lstlisting}%\pause
Für nummerierte Auflistungen kann man die Umgebung \textbf{enumerate} verwenden. Sie kann ebenfalls bis zu vier Ebenen tief geschachtelt werden.
\begin{lstlisting}[style=tex]
\begin{enumerate}
\item Stichpunkt 1
\item Stichpunkt 2
...
\end{enumerate}
\end{lstlisting}
\end{frame}

\begin{frame}[fragile]{Auflistung \& Aufzählung Nummerierung}
Die Aufzählungszeichen lassen sich auch verändern.
Dazu bindet man das Paket \textbf{enumerate} ein:
\begin{lstlisting}[style=tex]
\usepackage{enumerate}

\begin{document}
   \begin{enumerate}[a)]
   \item erste Ebene
      \begin{enumerate}[(i)]
         \item zweite Ebene mit der Einstellung [(i)]
      \end{enumerate}
   \end{enumerate}
\end{document}
\end{lstlisting}%\pause

\bigskip
Und so sieht es aus:
\begin{enumerate}[a)]
\item erste Ebene
\begin{enumerate}[(i)]
\item zweite Ebene mit der Einstellung [(i)]
\end{enumerate}
\end{enumerate}
\end{frame}

% ===== Tabulars =====
\subsection{Tabellen}
\begin{frame}[fragile]{Tabellen}
Einfache Tabellen können mit der Umgebung \textbf{tabular} erstellt werden.
\begin{lstlisting}[style=tex]
\begin{tabular}{Spaltendefinitionen}
Tabelleninhalt
\end{tabular}
\end{lstlisting}

In das Feld \{Spaltendefinitionen\} wird für jede Spalte die Formatierung angegeben. \\

\begin{itemize}
	\item \textbf{l} -- linksbündige Spalte (ohne Zeilenumbruch!) 
	\item \textbf{c} -- zentrierte Spalte (ohne Zeilenumbruch!)
	\item \textbf{r} -- rechtsbündige Spalte (ohne Zeilenumbruch!)
	\item \textbf{p\{Länge\}} -- entspricht der Definition von \texttt{\textbackslash parbox[c]\{Länge\}}, welche grundsätzlich im Blocksatz und mit Zeilenumbrüchen gesetzt wird
	\item \textbf{|} -- senkrechte Linie, kann beliebig erweitert werden zu ||...
\end{itemize}
%\begin{itemize}
%\item \textbf{c} -- für zentrierten Text (einzeilig)
%\item \textbf{l} -- für linksbündigen Text (einzeilig)
%\item \textbf{r} -- für rechtsbündigen Text (einzeilig)
%\item \textbf{p\{SIZEcm\}} -- Spalte soll SIZE cm breit sein und der Zelleninhalt wird automatisch umgebrochen
%\item \textbf{|} -- zieht eine vertikale Trennlinie zwischen zwei Spalten
%\end{itemize}
\end{frame}

\begin{frame}[fragile]{Tabellen}
Der Tabelleninhalt wird wie folgt gesetzt:
\begin{itemize}
\item \textbf{\&} -- trennt die Spalten voneinander
\item \textbf{\textbackslash\textbackslash} -- trennt die Zeilen voneinander
\item \textbf{\textbackslash hline} -- zieht eine horizontale Trennlinie zwischen zwei Zeilen
\end{itemize}

Beispiel:
\begin{lstlisting}[style=tex]
\begin{tabular}{| l | c | r |}
Spalte 1 & Spalte 2 & Spalte 3 \\
\hline  % horizontale Trennlinie
1 & 2 & 3 \\
\end{tabular}
\end{lstlisting}

\bigskip
Und so sieht es aus:
%\begin{center}
\renewcommand{\arraystretch}{1.2}
\begin{tabular}{| l | c | r |}
Spalte 1 & Spalte 2 & Spalte 3 \\
\hline
1 & 2 & 3 \\
\end{tabular}
%\end{center}

\bigskip
Dabei richtet sich die Tabellenbreite nach dem Inhalt.
\end{frame}

\begin{frame}[fragile]{Tabellen}
Möchte man, dass die Tabelle eine feste Breite hat und Zelleninhalt automatisch umgebrochen wird, benutzt man \textbf{tabularx}
\begin{lstlisting}[style=tex]
\usepackage{tabularx}

\begin{document}
   \begin{tabularx}{\linewidth}{| X | X | p{2cm} |}
   \hline
   ... & ... & ... \\
   \hline
   \end{tabularx}
\end{document}
\end{lstlisting}%\pause

So sieht es aus:\vspace{3pt}
\begin{tabularx}{\linewidth}{| X | X | p{2cm} |}
\hline
Diese Spalte ist genauso breit wie die rechts neben ihr stehende. & Der Text wird in Blocksatz gesetzt.  & Diese Spalte hat eine feste Breite von 2\,cm. \\
\hline
\end{tabularx}

\bigskip
Die Gesamtbreite wird zu gleichen Teilen auf alle Spalten, die mit \textbf{X} deklariert wurden, verteilt.
\end{frame}

%\begin{frame}[fragile]{Tabellen}
%Zeilen und Spalten zu einer verschmelzen:

%\begin{lstlisting}[style=tex]
%\usepackage{multirow}   % für multirow in tabulars
%\end{lstlisting}%\pause

%Beispiel:
%\begin{lstlisting}[style=tex]
%\begin{tabular}{|c|c|c|}
%  \hline
%  \multicolumn{2}{|c|}{Kuchen} & Hefe \\
%  \hline
%  Zeit & \multirow{2}{*}{zwei Zeilen} & Zucker \\
%  \cline{1-1}\cline{3-3}  % vertikale Trennlinie
%  Sieb & & Mehl \\
%  \hline
%\end{tabular}
%\end{lstlisting}

%\begin{center}
%So sieht es aus:
%\renewcommand{\arraystretch}{1.2}
%\begin{tabular}{|c|c|c|}
%\hline
%\multicolumn{2}{|c|}{eine Spalte} & Hefe \\
%\hline
%Zeit & \multirow{2}{*}{eine Zeile} & Zucker \\
%\cline{1-1}\cline{3-3}
%Sieb & & Mehl \\
%\hline
%\end{tabular}
%\end{center}
%\end{frame}

\begin{frame}[fragile]{Tabellenverzeichnis}
Mit \textbf{\texttt{\textbackslash listoftables}} kann ein Tabellenverzeichnis erstellt werden.
Die Tabellen müssen sich dafür in der \textbf{table}-Umgebung befinden.

\begin{lstlisting}[style=tex]
\begin{table}
   \caption{Grundlagen \LaTeX{} für Seminararbeiten}
   \label{tab:anwesenheit}
   \begin{tabular}{|c|c|c|}
      \hline
      Teilnehmer & Anwesenheit \\
      \hline 
      ...   
   \end{tabular}
\end{table}
\end{lstlisting}\label{tab:anwesenheit}%\pause

\begin{itemize}
\item Der Befehl \textbf{\textbackslash listoftables} erstellt automatisch das Tabellenverzeichnis.
\item Unsere Tabelle erzeugt den Eintrag ,,\textbf{Grundlagen \LaTeX{} für Seminararbeiten} S.~\ref{tab:anwesenheit}``.
\end{itemize}
\end{frame}

% ===== Spaltenlayout =====
%\subsection{Spaltenlayout}
%\begin{frame}[fragile]{Spaltenlayout}
%\begin{itemize}
%\item Einzelne Bereiche einer Seite können in mehrere Spalten aufgeteilt werden.
%\item Dafür wird das \textbf{multicol}-Paket benötigt.
%\end{itemize}

%\begin{lstlisting}[style=tex]
%\usepackage{multicol}

%\begin{document}
%   \begin{multicols}{ANZAHL}
%   Inhalt
%   \end{multicols}
%   ...
%\end{document}
%\end{lstlisting}

%{\itshape
%\begin{multicols}{3}
%Der Inhalt (z.B. Text und Bilder) werden auf die Anzahl der Spalten gleichmäßig verteilt. Satzart: linksbündig. Faustregel: Je schmaler das Layout, desto weniger Spalten sollte man nehmen. Der Spalten-Zwischenraum wird automatisch angepasst.
%\end{multicols}}
%\end{frame}

% ===== Minipages =====
\subsection{Minipages}
\begin{frame}[fragile]{Minipages}
Minipages geben neben Spalten und Tabellen die Möglichkeit, Inhalte zusammengehörig in einer Art \textbf{Container} zu anderen Inhalten auszurichten und ihnen eine feste Breite zu geben.
\begin{lstlisting}[style=tex]
\begin{minipage}[ÄUSSERE POSITION][HÖHE][INNERE POSITION]{BREITE}
Inhalt
\end{minipage}\end{lstlisting}
\begin{itemize}
\item ÄUSSERE POSITION richtet die Minipage relativ zur aktuellen Grundlinie aus:
\begin{itemize}
\item c = center
\item t = top
\item b = bottom \newline
\end{itemize}
\item HÖHE, ist eine gültige Längenangabe, durch die die Gesamthöhe der Minipage bestimmt wird. \newline
\item INNERE POSITION, richtet den Inhalt der Minipage innerhalb der angegebenen HÖHE aus.
\end{itemize}
\end{frame}

\begin{frame}[fragile]{Minipages}
Beispiel:
\begin{lstlisting}[style=tex]
\begin{minipage}{0.6\linewidth}
   Inhalt 1
\end{minipage}
\hfill % schiebt die nachfolgende Minipage an den rechten Layoutrand
\begin{minipage}{0.3\linewidth}
   Inhalt 2
\end{minipage}
\end{lstlisting}

\begin{minipage}{0.7\linewidth}
\itshape Die Minipages richten sich standardmä{\ss}ig vertikal mittig zueinander aus. Der Zwischenraum beträgt im Beispielcode $10\%$ der Layoutbreite.
\end{minipage}
\hfill 
\begin{minipage}{0.25\linewidth}
   \includegraphics[width=\linewidth]{img/intro}
\end{minipage}
\end{frame}
%\section{Aufbau eines Dokuments}

% ////////////////// Dokumentenklassen //////////////
\subsection{Dokumententypen}
\begin{frame}[fragile]{Dokumentenklassen}
\begin{itemize}
\item \LaTeX{} bringt standardmäßig verschiedene Grunddokumententypen mit, die \textbf{verschiedene Eigenschaften} haben \newline
\item in jedem Dokument muss die \textbf{Dokumentenklasse} angegeben werden $\Rightarrow$ erster Befehl eines Dokuments \newline
\item Befehl zur Angabe der Klasse beinhaltet die Dokumentenklasse (in geschweiften Klammern) sowie die Angabe möglicher \textbf{Optionen} (in eckigen Klammern)
\end{itemize}
%\begin{varwidth}{0.45\linewidth}
\begin{lstlisting}[style=tex]
\documentclass[Optionen]{Dokumentenklasse}\end{lstlisting}
\begin{itemize}
\item \textbf{Übersicht:} \url{http://www.kkittel.de/wiki/doku.php?id=grundlegende_einstellungen:dokumentenklassen}
\end{itemize}
\end{frame}


% ////////////////// Typen Übersicht //////////////
\begin{frame}{Dokumententypen}
\begin{itemize}
\item \textbf{Article:} geeignet für kurze technische Artikel \textit{(scrartcl)}
\item \textbf{Report:} geeignet für längere technische Artikel \textit{(scrreprt)}
\item \textbf{Book:} Drucklayout standardmäßig zweiseitig; z.B.: automatische Kopfzeile mit Seitenzahl \textit{(scrbook)}

\item \textbf{Letter:} geeignet zum Schreiben von Briefen \textit{(scrlttr2)}
\end{itemize}
\end{frame}

% ///////////// Pakete ///////////////
\subsection{Pakete}
\begin{frame}[fragile]{Pakete}
\TeX{} und \LaTeX{} besitzen einen Grundstock an Befehlen. Wenn man darüber hinaus Modifizierungen vornehmen möchte, muss man zusätzliche Pakete einbinden, die weitere Befehle zur Verfügung stellen.\pause

\begin{lstlisting}[style=tex]
\usepackage{fontspec}
    % Schriftpaket, funktioniert nur mit den neueren Compilern z.B. XeLaTeX
\usepackage{microtype}
    % verbessert die Worttrennung
\usepackage[ngerman]{babel}
    % Spracheinstellung: richtige Silbentrennung.
\usepackage{lmodern}
    % verändert verwendete Schriftart, damit sie weniger pixelig ist
\end{lstlisting}
\end{frame}

% ////////////// Begin Dokument //////////////
\subsection{Das Dokument}
\begin{frame}[fragile]{Das Dokument}
Im Kopf des Dokuments kann man Einstellungen vornehmen und zusätzliche Pakete einbinden. Der eigentliche Inhalt folgt dann in der \textbf{document}-Umgebung:
\begin{lstlisting}[style=tex]
\documentclass[paper=a4, 10pt, ngerman]{scrartcl}

\begin{document}
   Mein Inhalt
\end{document}\end{lstlisting}
\end{frame}

% //////// Titlepage //////////////
\subsection{Titelseite}
\begin{frame}[fragile]{Titelseite}
Die für ein Dokument wichtige Angaben (wie z.B. Titel, Autor und Datum) lassen sich sehr einfach erstellen und werden dann automatisch vorformatiert.
\begin{lstlisting}[style=tex]
\begin{titlepage}
  \title{Crashkurs LaTeX}
  \subtitle{Mentoring WiSe 2016}
  \subject{Computer Science}
  \author{Anja Wolffgramm, Diane Hanke}
  \date{25. November 2016}
\end{titlepage}

\maketitle % erstellt die Titelseite
\end{lstlisting}

% \textbf{Anwendung:}
% \vspace{-2ex}
% \begin{lstlisting}[style=tex]
% \begin{titlepage}
%   ...
% \end{titlepage}

% \begin{document}
%    \maketitle
% \end{document}
% \end{lstlisting}
% Wendet man die Zeilen 1 -- 7 vor Beginn des Dokuments an, wird eine extra Cover-Seite erstellt.

% \begin{lstlisting}[style=tex]
% \begin{document}
%   \begin{titlepage}
%     ...
%   \end{titlepage}
  
%   \maketitle
% \end{document}
% \end{lstlisting}
% Wendet man es innerhalb des Dokuments an, wird nachfolgender Inhalt auf die selbe Seite gesetzt.
\end{frame}

%\section{Befehle \& Umgebungen}

% ===== Makros =====
\begin{frame}[fragile]{Makros}
%\begin{multicols}{2}%  erstellt 2 Spalten
Ein Makro wird mit einem Backslash eingeleitet:

\begin{lstlisting}[style=tex]
\huge nachfolgender Text
\end{lstlisting}

%\pause 
Und so sieht es aus: {\huge nachfolgender Text}

\vspace{2ex}
\pause Manche Makros erhalten Argumente in geschweiften Klammern:
\begin{lstlisting}[style=tex]
\textcolor{Pink}{mein pinker Unicorn-Text}
\end{lstlisting}

%\pause 
Und so sieht es aus: \textcolor{Pink}{mein pinker Unicorn-Text}
%\end{multicols}
\end{frame}

% ////////////////// Umgebung //////////////
\begin{frame}[fragile]{Umgebungen}
Eine Umgebung fasst einen Bereich ein. Sie hat immer die folgende Form:

\begin{lstlisting}[style=tex]
\begin{meineUmgebung}
Teil dazwischen
\end{meineUmgebung}\end{lstlisting}

\pause Als Beispiel dient da die Dokument-Umgebung:

\begin{lstlisting}[style=tex]
\begin{document}
mein cooles LaTeX-Dokument
\end{document}\end{lstlisting}
\end{frame}
%\section{Textstrukturierung}

% ===== Gliederung =====
\subsection{Gliederung \& Inhaltsverzeichnis}
\begin{frame}[fragile]{Gliederung \& Inhaltsverzeichnis}
%\begin{varwidth}{0.45\linewidth}
%Eine Überschrift wird mit folgendem Befehl erstellt:
Mit dem Makro \texttt{\textbackslash section} wird eine Überschrift in der ersten Gliederungsebene erzeugt:
\begin{lstlisting}[style=tex]
\section{head}
\end{lstlisting}
%\pause
Für eine Überschrift in der zweiten Gliederungsebene benutzen wir das Makro \texttt{\textbackslash subsection}:
\begin{lstlisting}[style=tex]
\subsection{subhead}
\end{lstlisting}
%\pause
In der dritten Gliederungsebene benutzen wir das Makro \texttt{\textbackslash subsubsection} für eine Überschrift:
\begin{lstlisting}[style=tex]
\subsubsection{subsubhead}
\end{lstlisting}
\end{frame}

% ===== Inhaltsverzeichnis =====
\begin{frame}[fragile]{Gliederung \& Inhaltsverzeichnis}
Das Inhaltsverzeichnis wird automatisch mit dem Makro
\begin{lstlisting}[style=tex]
\tableofcontents
\end{lstlisting}
erstellt. Dabei werden alle Überschriften übernommen.
%\pause
\newline
\newline
Möchte man nicht, dass eine Überschrift im Verzeichnis auftaucht, benutzt man das Sternchen:
\begin{lstlisting}[style=tex]
\section*{diese Überschrift taucht nicht auf}
\end{lstlisting}
\end{frame}

% ===== Absätze, Zeilenumbrüche =====
\subsection{Absätze \& Zeilenumbrüche}
\begin{frame}[fragile]{Absätze \& Zeilenumbrüche}

Ein Zeilenumbruch kann entweder durch eine \textbf{Leerzeile} im Quelltext oder durch das Makro \textbf{\textbackslash\texttt{par}}  erreicht werden.

\begin{itemize}
	\item Gliederung des Inhalts in Gedankengänge
	\item 1. Zeile ist eingerückt
	\item Abstand zwischen zwei Absätzen
\end{itemize}

%\begin{multicols}{2}
%\begin{lstlisting}[style=tex]
%\par
%\end{lstlisting}
%\begin{lstlisting}[style=tex]
%\newline
%\end{lstlisting}
%\begin{lstlisting}[style=tex]
%\linebreak
%\end{lstlisting}
%\begin{lstlisting}[style=tex]
%\\
%\end{lstlisting}
%\end{multicols}%\pause

Möchte man nicht, dass die erste Zeile eines Absatzes eingerückt wird, kann man folgende Zeile in die Präambel des Dokuments vornehmen:
\begin{lstlisting}[style=tex]
\parindent 0pt
\end{lstlisting}

Einen manuellen Zeilenumbruch ohne Abstand kann man mit folgenden Makros erzeugen:
\begin{lstlisting}[style=tex]
\newline              % Zeilenumbruch im selben Absatz
\linebreak
\\
\end{lstlisting}

\end{frame}

% ===== Auflistung, Aufzählung =====
\subsection{Auflistung \& Aufzählung}
\begin{frame}[fragile]{Auflistung \& Aufzählung}
Für unnummerierte Auflistungen kann man die Umgebung \textbf{itemize} benutzen. Sie kann bis zu vier Ebenen tief geschachtelt werden.
\begin{lstlisting}[style=tex]
\begin{itemize}
\item Stichpunkt 1
\item Stichpunkt 2
...
\end{itemize}
\end{lstlisting}%\pause
Für nummerierte Auflistungen kann man die Umgebung \textbf{enumerate} verwenden. Sie kann ebenfalls bis zu vier Ebenen tief geschachtelt werden.
\begin{lstlisting}[style=tex]
\begin{enumerate}
\item Stichpunkt 1
\item Stichpunkt 2
...
\end{enumerate}
\end{lstlisting}
\end{frame}

\begin{frame}[fragile]{Auflistung \& Aufzählung Nummerierung}
Die Aufzählungszeichen lassen sich auch verändern.
Dazu bindet man das Paket \textbf{enumerate} ein:
\begin{lstlisting}[style=tex]
\usepackage{enumerate}

\begin{document}
   \begin{enumerate}[a)]
   \item erste Ebene
      \begin{enumerate}[(i)]
         \item zweite Ebene mit der Einstellung [(i)]
      \end{enumerate}
   \end{enumerate}
\end{document}
\end{lstlisting}%\pause

\bigskip
Und so sieht es aus:
\begin{enumerate}[a)]
\item erste Ebene
\begin{enumerate}[(i)]
\item zweite Ebene mit der Einstellung [(i)]
\end{enumerate}
\end{enumerate}
\end{frame}

% ===== Tabulars =====
\subsection{Tabellen}
\begin{frame}[fragile]{Tabellen}
Einfache Tabellen können mit der Umgebung \textbf{tabular} erstellt werden.
\begin{lstlisting}[style=tex]
\begin{tabular}{Spaltendefinitionen}
Tabelleninhalt
\end{tabular}
\end{lstlisting}

In das Feld \{Spaltendefinitionen\} wird für jede Spalte die Formatierung angegeben. \\

\begin{itemize}
	\item \textbf{l} -- linksbündige Spalte (ohne Zeilenumbruch!) 
	\item \textbf{c} -- zentrierte Spalte (ohne Zeilenumbruch!)
	\item \textbf{r} -- rechtsbündige Spalte (ohne Zeilenumbruch!)
	\item \textbf{p\{Länge\}} -- entspricht der Definition von \texttt{\textbackslash parbox[c]\{Länge\}}, welche grundsätzlich im Blocksatz und mit Zeilenumbrüchen gesetzt wird
	\item \textbf{|} -- senkrechte Linie, kann beliebig erweitert werden zu ||...
\end{itemize}
%\begin{itemize}
%\item \textbf{c} -- für zentrierten Text (einzeilig)
%\item \textbf{l} -- für linksbündigen Text (einzeilig)
%\item \textbf{r} -- für rechtsbündigen Text (einzeilig)
%\item \textbf{p\{SIZEcm\}} -- Spalte soll SIZE cm breit sein und der Zelleninhalt wird automatisch umgebrochen
%\item \textbf{|} -- zieht eine vertikale Trennlinie zwischen zwei Spalten
%\end{itemize}
\end{frame}

\begin{frame}[fragile]{Tabellen}
Der Tabelleninhalt wird wie folgt gesetzt:
\begin{itemize}
\item \textbf{\&} -- trennt die Spalten voneinander
\item \textbf{\textbackslash\textbackslash} -- trennt die Zeilen voneinander
\item \textbf{\textbackslash hline} -- zieht eine horizontale Trennlinie zwischen zwei Zeilen
\end{itemize}

Beispiel:
\begin{lstlisting}[style=tex]
\begin{tabular}{| l | c | r |}
Spalte 1 & Spalte 2 & Spalte 3 \\
\hline  % horizontale Trennlinie
1 & 2 & 3 \\
\end{tabular}
\end{lstlisting}

\bigskip
Und so sieht es aus:
%\begin{center}
\renewcommand{\arraystretch}{1.2}
\begin{tabular}{| l | c | r |}
Spalte 1 & Spalte 2 & Spalte 3 \\
\hline
1 & 2 & 3 \\
\end{tabular}
%\end{center}

\bigskip
Dabei richtet sich die Tabellenbreite nach dem Inhalt.
\end{frame}

\begin{frame}[fragile]{Tabellen}
Möchte man, dass die Tabelle eine feste Breite hat und Zelleninhalt automatisch umgebrochen wird, benutzt man \textbf{tabularx}
\begin{lstlisting}[style=tex]
\usepackage{tabularx}

\begin{document}
   \begin{tabularx}{\linewidth}{| X | X | p{2cm} |}
   \hline
   ... & ... & ... \\
   \hline
   \end{tabularx}
\end{document}
\end{lstlisting}%\pause

So sieht es aus:\vspace{3pt}
\begin{tabularx}{\linewidth}{| X | X | p{2cm} |}
\hline
Diese Spalte ist genauso breit wie die rechts neben ihr stehende. & Der Text wird in Blocksatz gesetzt.  & Diese Spalte hat eine feste Breite von 2\,cm. \\
\hline
\end{tabularx}

\bigskip
Die Gesamtbreite wird zu gleichen Teilen auf alle Spalten, die mit \textbf{X} deklariert wurden, verteilt.
\end{frame}

%\begin{frame}[fragile]{Tabellen}
%Zeilen und Spalten zu einer verschmelzen:

%\begin{lstlisting}[style=tex]
%\usepackage{multirow}   % für multirow in tabulars
%\end{lstlisting}%\pause

%Beispiel:
%\begin{lstlisting}[style=tex]
%\begin{tabular}{|c|c|c|}
%  \hline
%  \multicolumn{2}{|c|}{Kuchen} & Hefe \\
%  \hline
%  Zeit & \multirow{2}{*}{zwei Zeilen} & Zucker \\
%  \cline{1-1}\cline{3-3}  % vertikale Trennlinie
%  Sieb & & Mehl \\
%  \hline
%\end{tabular}
%\end{lstlisting}

%\begin{center}
%So sieht es aus:
%\renewcommand{\arraystretch}{1.2}
%\begin{tabular}{|c|c|c|}
%\hline
%\multicolumn{2}{|c|}{eine Spalte} & Hefe \\
%\hline
%Zeit & \multirow{2}{*}{eine Zeile} & Zucker \\
%\cline{1-1}\cline{3-3}
%Sieb & & Mehl \\
%\hline
%\end{tabular}
%\end{center}
%\end{frame}

\begin{frame}[fragile]{Tabellenverzeichnis}
Mit \textbf{\texttt{\textbackslash listoftables}} kann ein Tabellenverzeichnis erstellt werden.
Die Tabellen müssen sich dafür in der \textbf{table}-Umgebung befinden.

\begin{lstlisting}[style=tex]
\begin{table}
   \caption{Grundlagen \LaTeX{} für Seminararbeiten}
   \label{tab:anwesenheit}
   \begin{tabular}{|c|c|c|}
      \hline
      Teilnehmer & Anwesenheit \\
      \hline 
      ...   
   \end{tabular}
\end{table}
\end{lstlisting}\label{tab:anwesenheit}%\pause

\begin{itemize}
\item Der Befehl \textbf{\textbackslash listoftables} erstellt automatisch das Tabellenverzeichnis.
\item Unsere Tabelle erzeugt den Eintrag ,,\textbf{Grundlagen \LaTeX{} für Seminararbeiten} S.~\ref{tab:anwesenheit}``.
\end{itemize}
\end{frame}

% ===== Spaltenlayout =====
%\subsection{Spaltenlayout}
%\begin{frame}[fragile]{Spaltenlayout}
%\begin{itemize}
%\item Einzelne Bereiche einer Seite können in mehrere Spalten aufgeteilt werden.
%\item Dafür wird das \textbf{multicol}-Paket benötigt.
%\end{itemize}

%\begin{lstlisting}[style=tex]
%\usepackage{multicol}

%\begin{document}
%   \begin{multicols}{ANZAHL}
%   Inhalt
%   \end{multicols}
%   ...
%\end{document}
%\end{lstlisting}

%{\itshape
%\begin{multicols}{3}
%Der Inhalt (z.B. Text und Bilder) werden auf die Anzahl der Spalten gleichmäßig verteilt. Satzart: linksbündig. Faustregel: Je schmaler das Layout, desto weniger Spalten sollte man nehmen. Der Spalten-Zwischenraum wird automatisch angepasst.
%\end{multicols}}
%\end{frame}

% ===== Minipages =====
\subsection{Minipages}
\begin{frame}[fragile]{Minipages}
Minipages geben neben Spalten und Tabellen die Möglichkeit, Inhalte zusammengehörig in einer Art \textbf{Container} zu anderen Inhalten auszurichten und ihnen eine feste Breite zu geben.
\begin{lstlisting}[style=tex]
\begin{minipage}[ÄUSSERE POSITION][HÖHE][INNERE POSITION]{BREITE}
Inhalt
\end{minipage}\end{lstlisting}
\begin{itemize}
\item ÄUSSERE POSITION richtet die Minipage relativ zur aktuellen Grundlinie aus:
\begin{itemize}
\item c = center
\item t = top
\item b = bottom \newline
\end{itemize}
\item HÖHE, ist eine gültige Längenangabe, durch die die Gesamthöhe der Minipage bestimmt wird. \newline
\item INNERE POSITION, richtet den Inhalt der Minipage innerhalb der angegebenen HÖHE aus.
\end{itemize}
\end{frame}

\begin{frame}[fragile]{Minipages}
Beispiel:
\begin{lstlisting}[style=tex]
\begin{minipage}{0.6\linewidth}
   Inhalt 1
\end{minipage}
\hfill % schiebt die nachfolgende Minipage an den rechten Layoutrand
\begin{minipage}{0.3\linewidth}
   Inhalt 2
\end{minipage}
\end{lstlisting}

\begin{minipage}{0.7\linewidth}
\itshape Die Minipages richten sich standardmä{\ss}ig vertikal mittig zueinander aus. Der Zwischenraum beträgt im Beispielcode $10\%$ der Layoutbreite.
\end{minipage}
\hfill 
\begin{minipage}{0.25\linewidth}
   \includegraphics[width=\linewidth]{img/intro}
\end{minipage}
\end{frame}
%\input{slides/pause} % Pause ----------------
%\section{Abbildungen}
% ===== Bilder einbinden =====
\subsection{Einbinden von Bildern}
\begin{frame}[fragile]{Bilder}
\begin{minipage}{0.6\linewidth}
Für Bilder wird das Paket \texttt{graphicx} benötigt:
\begin{lstlisting}[style=tex]
\usepackage{graphicx}
\end{lstlisting}

Ein Bild bindet man mit folgendem Code ein:

\begin{lstlisting}[style=tex]
\includegraphics[OPTION]{path/name}
\end{lstlisting}

%\pause
Options:
\begin{itemize}
\item \verb|scale=1| (Werte $>0$)
\item \verb|width=0.5\linewidth|
\item \verb|keepaspectratio=true|
\end{itemize}

%\pause
Beispiel:
\begin{lstlisting}[style=tex]
\includegraphics[width=\linewidth]%
   {img/intro}
   \end{lstlisting}
\end{minipage}
\hfill
\begin{minipage}{0.35\linewidth}
%\pause
\includegraphics[width=\linewidth]{img/intro}
\end{minipage}
\end{frame}

% ===== Bildumgebung =====
\subsection{Bildumgebung}
\begin{frame}[fragile]{Bildumgebung}   
\begin{multicols}{2}
	Die Bildumgebung \texttt{figure} bringt folgende Vorteile mit:
\begin{itemize}
\item Bildbeschreibung
\item Zentrierung
\item Aufnahme in das Abbildungsverzeichnis
%\item Bildquelle: \url{https://eu.fotolia.com/Content/Comp/82385255}
\end{itemize}

\begin{figure}
\centering
\includegraphics[width=0.45\linewidth]{img/intro}
\caption[awesome_image]{Bildbeschreibung}
\label{fig:awesome_image}
\end{figure}
\end{multicols}
\vspace{-2ex}
\begin{lstlisting}[style=tex]
\begin{figure}
   \centering
   \includegraphics[width=0.5\linewidth]{img/intro}
   \caption[awesome_image]{Bildbeschreibung}
   \label{fig:awesome_image}
\end{figure}\end{lstlisting}
\end{frame}

% ===== Abbildungsverzeichnis =====
\subsection{Abbildungsverzeichnis}
\begin{frame}[fragile]{Abbildungsverzeichnis}
Ein Abbildungsverzeichnis kann man mit dem folgenden Makro erstellen:

\begin{lstlisting}[style=tex]
\listoffigures
\end{lstlisting}

%\listoffigures
\end{frame}
%\section{Mathematikmodus}

% ===== Mathematikmodus =====
\begin{frame}[fragile]{Mathematikmodus}
Der Mathematikmodus ist eine Umgebung, die dafür optimiert ist, mathematische Formeln und Symbole darzustellen. Dafür werden folgende Pakete benötigt:
\begin{lstlisting}[style=tex, mathescape]
\usepackage{amsmath}    % MUSS vor fontspec geladen werden
\usepackage{mathtools}   % modifiziert amsmath
\usepackage{amssymb}    % mathematische symbole, für \checkmarks
\usepackage{amsthm}      % für proof
\usepackage{mathrsfs}     % für \mathscr
\usepackage{latexsym}
\usepackage{marvosym}   % zusätzliche Zeichen, z.B. Lightning
\usepackage{cancel}         % für das Durchstreichen u.a. in Matheformeln mit \cancel
\end{lstlisting}
\end{frame}

% \begin{frame}[fragile]{Mathematikmodus}
% \begin{itemize}
% \item zahlreiche Befehle, Symbole und Umgebungen funktionieren nur in diesem Modus
% \item andere funktionieren gerade in diesem Modus nicht und man muss für diese entsprechend andere Modi oder Befehle verwenden:

% \vspace{1ex}
% \begin{tabular}{l|l|l}
% Befehl & Textmodus & Mathematikmodus \\
% \hline
% Unterstrich & \verb|\_| & \_ \\
% Dach & \verb|\^{}| & \verb|^| \\
% Tabelle & \verb|tabular| & \verb|array| \\
% Bold & \verb|\textbf{}| & \verb|\mathbf{}| \\
% Backslash & \verb|\textbackslash| & \verb|\backslash| \\
% \end{tabular}
% \end{itemize}
% \end{frame}

\subsection{Zeilenmodus vs. Absatzmodus}
\begin{frame}[fragile]{Zeilenmodus vs. Absatzmodus}	
Grundsätzlich unterscheidet schon \TeX{} den Zeilenmodus (inline) und den Absatzmodus (abgesetzten Modus). Unter dem Zeilenmodus versteht man das Einfügen von mathematischen Elementen in die laufende Zeile und unter dem abgesetzten Modus das Setzen einer Gleichung als eigenständigen Absatz.
%\item Im Fließtext wird der betreffende Ausdruck mit Dollarzeichen \textbf{\$} umgeben. z.B.: \verb|$E = mc^2$|
% \begin{lstlisting}[style=tex]
% Mitten im $E = mc^2$ Text...
% \end{lstlisting}

%\vspace{0.5ex} Mitten im Text steht die Formel $E = mc^2$ von Einstein.\pause

%\item Um einzeilige Formeln abzusetzen, baut man eine Umgebung auf, innerhalb derer der Mathematikmodus eingeschaltet ist:

%\begin{lstlisting}[style=tex]
%\[ E=mc^2 \]
%\end{lstlisting}
%Führt zu:
%\[ E=mc^2 \]
%\end{itemize}
\end{frame}

\begin{frame}[fragile]{Zeilenmodus}
Der Zeilenmodus kann durch drei verschiedene Umgebungen aktiviert werden:
\begin{lstlisting}[mathescape=true]{style=tex}
\(\sum_{i=1}^{n}i=\frac{1}{2}n\cdot (n+1)\)
\end{lstlisting}
Führt zu: \(\sum_{i=1}^{n}i=\frac{1}{2}n\cdot (n+1)\)
\begin{lstlisting}[mathescape=false]{style=tex}
$\sum_{i=1}^{n}i=\frac{1}{2}n\cdot (n+1)$
\end{lstlisting}
Führt zu: $\sum_{i=1}^{n}i=\frac{1}{2}n\cdot (n+1)$
\begin{lstlisting}[mathescape=false]{style=tex}
\begin{math}
\sum_{i=1}^{n}i=\frac{1}{2}n\cdot (n+1)
\end{math}
\end{lstlisting}
Führt zu:
\begin{math}
\sum_{i=1}^{n}i=\frac{1}{2}n\cdot (n+1)
\end{math}
\end{frame}

\begin{frame}[fragile]{Absatzmodus}
Die Umgebung \texttt{align} schaltet den Mathematikmodus ein. Die Formel wird zentriert und es werden Zeilenumbrüche durch das Makro \texttt{\textbackslash\textbackslash} erlaubt. Die Umgebung nummeriert die Zeilen und ermöglicht ein Ausrichten der Zeilen mit dem Makro \texttt{\&} zueinander. 
\begin{lstlisting}[style=tex]
\begin{align}
\sum_{i=0}^{n+1} i &= \sum_{i=0}^n i + (n+1) \\
                                    &= \frac{n(n+1)}{2} + (n+1)
\end{align}
\end{lstlisting}

%\pause 
Und so sieht es aus:
\begin{align}
\sum_{i=0}^{n+1} i &= \sum_{i=0}^n i + (n+1) \\
                       &= \frac{n(n+1)}{2} + (n+1)
\end{align}%\pause

Will man keine Nummerierung, nutzt man \textbf{align*}
\end{frame}

\subsection{Spezielle Formatierungen}
\begin{frame}[fragile]{Spezielle Formatierungen}
\begin{minipage}{0.3\linewidth}
\begin{lstlisting}[style=tex]
\begin{theorem}
...
\end{theorem}
\end{lstlisting}
\end{minipage}
\hfill
\begin{minipage}{0.65\linewidth}
\begin{theorem}
This is a theorema about right triangles and can be summarised in the next 
equation 
\[ x^2 + y^2 = z^2 \]
\end{theorem}
\end{minipage}

\begin{minipage}{0.3\linewidth}
\begin{lstlisting}[style=tex]
\begin{corollary}
...
\end{corollary}
\end{lstlisting}
\end{minipage}
\hfill
\begin{minipage}{0.65\linewidth}
\begin{corollary}
There's no right rectangle whose sides measure 3cm, 4cm, and 6cm.
\end{corollary}
\end{minipage}

\begin{minipage}{0.3\linewidth}
\begin{lstlisting}[style=tex]
\begin{lemma}
...
\end{lemma}
\end{lstlisting}
\end{minipage}
\hfill
\begin{minipage}{0.65\linewidth}
\begin{lemma}
Given two line segments whose lengths are $a$ and $b$ respectively there is a 
real number $r$ such that $b=r\cdot a$.
\end{lemma}
\end{minipage}
\end{frame}

\begin{frame}[fragile]{Spezielle Formatierungen}
\begin{minipage}{0.3\linewidth}
\begin{lstlisting}[style=tex]
\begin{proof}
...
\end{proof}
\end{lstlisting}
\end{minipage}
\hfill
\begin{minipage}{0.65\linewidth}
\begin{proof}
Mit $r=\frac{b}{a}$ gilt stets $b=r\cdot a$.
\end{proof}
\end{minipage}

\begin{minipage}{0.3\linewidth}
\begin{lstlisting}[style=tex]
\begin{definition}
...
\end{definition}
\end{lstlisting}
\end{minipage}
\hfill
\begin{minipage}{0.65\linewidth}
\begin{definition}
\[ fak(n) := \begin{cases}
1, & n = 0 \\
n\cdot fak(n-1), & \text{sonst}
\end{cases}
\quad
\forall n \in \mathbb{N} \]
\end{definition}
\end{minipage}
\end{frame}
%\input{slides/farben} 
%\section{Querverweise \& Hyperlinks}
\begin{frame}[fragile]{Querverweise \& Hyperlinks}
Um Querverweise im Dokument zu setzen, muss man folgende Pakete einbinden:
\begin{lstlisting}[style=tex]
\usepackage[        % Verlinkungen
  colorlinks,            % farbige Schrift, statt farbiger Rahmen
  linktocpage,        % verlinkt im Abb.Verzeichnis Seitenzahl statt Bildunterschrift
  linkcolor=blue       % setzt Farbe der Links auf blau
]{hyperref}
\usepackage{url}  % für Webadressen: "\url{http://www.example.com}"
\end{lstlisting}\pause

\bigskip
Einen dokumentinternen Verweis\label{LABELNAME} kann man wie folgt erstellen:
\begin{lstlisting}[style=tex]
\label{LABELNAME} % setzt einen Verweis an die entsprechende Stelle.
\end{lstlisting}

Auf diesen kann in einer anderen Stelle referenziert werden:
\begin{lstlisting}[style=tex]
Hier steht ein Verweis zur Seite \ref{LABELNAME}.
\end{lstlisting}\pause

\bigskip
$\Rightarrow$ Hier steht ein Verweis zur Seite \ref{LABELNAME}.
\end{frame}

\begin{frame}[fragile]{Querverweise \& Hyperlinks}
Auch \textbf{URL}s kann man einfach hinzufügen:
\begin{lstlisting}[style=tex]
\url{http://latex.org}
\end{lstlisting}
So sieht es aus: \url{http://latex.org}\pause

\bigskip
Ein \textbf{Verzeichnis der Querverweise} wird automatisch mit folgendem Befehl erstellt:
\begin{lstlisting}[style=tex]
\linktocpage
\end{lstlisting}
\end{frame}

%\input{slides/pause} % Pause ----------------
%\section{Quellcode einbinden}

% ////////////////// Verbatim //////////////
\subsection{Verbatim}
\begin{frame}[fragile]{Verbatim}
Wenn Text nicht vom \LaTeX-Compiler interpretiert werden soll, kann man ihn in ein \textbf{verb}-Befehl setzen:

\begin{lstlisting}[style=tex]
\verb| nicht interpretierter Text |
\end{lstlisting}
Dabei muss dem Befehl mitgeteilt werden, wann dies endet. Dies geschieht durch ein Zeichen, das den Bereich einleitet und abschlie{\ss}t.

%\pause
\vspace{2ex}
Für mehrere Zeilen gibt es die \textbf{verbatim}-Umgebung:

\begin{lstlisting}[style=tex]
\usepackage{verbatim}

\begin{document}
  \begin{verbatim}
  diese Zeilen
  sollen nicht interpretiert
  werden       _ & \ -- % Kommentar
  \end{verbatim}
\end{document}
\end{lstlisting}
\end{frame}

% ////////////////// Quellcode inline einbinden //////////////
\subsection{Lstlistings}
\begin{frame}[fragile]{Lstlistings: Quellcode inline einbinden}
%Um Programmcode in das Dokument einzubinden, möchte man sprachspezifisches \textbf{Syntax-Highlighting} und Zeilennummerierungen haben.
Um Programmcode in das Dokument einzubinden, kann man das Paket \textbf{\texttt{listings}} verwenden.
%\pause
\vspace{2ex}
%Dafür gibt es das \textbf{listings}-Paket:

\begin{lstlisting}[style=tex]
\usepackage{listings}

\begin{document}
  \begin{lstlisting}[language=python, caption=Beschriftung]
      % hier steht Python-Code
  \end{lstlitsing}
\end{document}\end{lstlisting}

%Und so sähe Haskell-Code aus:
%\begin{lstlisting}[style=hs, caption= Fakultät in Haskell]
%fak :: [Integer] -> [Integer]
%fak 0 = 1
%fak n = n * fak(n-1)
%\end{lstlisting}

So sieht dann Python-Code aus:
\begin{lstlisting}[language=python, caption = Berechnung der n-ten Fibonacci in Python durch Rekursion]
def fib_rek(n):
    if n <= 0:
        raise Exception
    elif n == 1 or n == 2:
        return 1
    else:
        return fib_rek(n-1) + fib_rek(n-2)
\end{lstlisting}

% \begin{lstlisting}[style=py, caption=Beschriftung]
% def myFunction():
%     # hier kann man wild coden
%     return True\end{lstlisting}
\end{frame}

% ////////////////// Quellcode extern einbinden //////////////
\begin{frame}[fragile]{Quellcode extern einbinden}
In der Regel programmiert man in einer Datei und möchte Teile dieser im \LaTeX-Dokument einbinden. Dies geht mittels \textbf{lstinputstring}:

\begin{lstlisting}[style=tex, caption=lstlisting]
\lstinputlisting[language=latex, caption={Beschriftung}]{pfad/filename.tex}\end{lstlisting}

%\pause 
Man kann auch nur ein paar Zeilen des Programmcodes einbinden:

\begin{lstlisting}[style=tex, firstnumber=2]
\lstinputlisting[language=latex, linerange=23-42, firstnumber=23]{pfad/filename.tex}\end{lstlisting}

\begin{lstlisting}[style=tex, firstnumber=3]
\lstinputlisting[language=latex, firstline=23, lastline=42]{pfad/filename.tex}\end{lstlisting}

\begin{itemize}
\item \textbf{linerange} -- Bereich im Quellcode
\item \textbf{firstline} -- Beginn des einzubindenden Quellcodes
\item \textbf{lastline} -- Ende des einzubindenden Quellcodes 
\item \textbf{firstnumber} -- Beginn der Zeilen-Nummerierung
\end{itemize}
\end{frame}

% ////////////////// Liste der Programmcodes //////////////
\subsection{Liste der Programmcodes}
\begin{frame}[fragile]
Eine Liste der \textbf{lstlistings} wird wie folgt erstellt:

\begin{lstlisting}[style=tex]
\lstlistoflistings\end{lstlisting}
\end{frame}


%\section{Eigene Kommandos und Umgebungen}

% ////////////////// Eigene Kommandos erstellen //////////////
\subsection{Eigene Kommandos erstellen}

\begin{frame}[fragile]{Eigene Kommandos erstellen}
Man kann sich eigene Kommandos erstellen, um z.\,B. Schreibaufwand einzusparen.

\begin{lstlisting}[style=tex]
\newcommand{\name}{Was es tun soll}\end{lstlisting}

\pause Hier ein Beispiel:
\begin{lstlisting}[style=tex]
\newcommand\zz{\ensuremath{\raisebox{+0.25ex}{Z}% zu-zeigen-Symbol
   \kern-0.4em\raisebox{-0.25ex}{Z}%
   \;\xspace}%
}\end{lstlisting}

\vspace{1ex}\pause Und so sieht es aus: $\zz$
\end{frame}

\begin{frame}[fragile]{Eigene Kommandos}
Es lassen sich auch Kommandos erstellen, welche Argumente erhalten und diese benutzen:

\begin{lstlisting}[style=tex]
\newcommand{\name}[Argumentzahl]{was es mit dem Argument #1 tun soll}\end{lstlisting}

\pause Hier ein Beispiel:

\begin{lstlisting}[style=tex]
\newcommand{\Quellcode}[3]{\lstinputlisting[%
   language=#2,   % 2. Argument: filename
   caption={#3}] % 3. Argument: Beschriftung
   {#1.#2}}         % 1. Argument: path\end{lstlisting}

\pause In der Anwendung:
\begin{lstlisting}[style=tex]
\Quellcode{src/myFile}{py}{Ein Python-Programm}\end{lstlisting}
\end{frame}

% ////////////////// Eigene Umgebungen erstellen //////////////
\subsection{Eigene Umgebungen erstellen}
\begin{frame}[fragile]{Eigene Umgebungen erstellen}
Möchte man eine bereits vorhandene Umgebung modifizieren, geht dies nicht mittels Kommando, sondern mit \verb|\newenvironment| .

\begin{lstlisting}[style=tex]
\newenvironment{name}[Argumentzahl]{Befehlsbeginn}{Befehlsende}\end{lstlisting}

\pause Nachfolgend ein Beispiel:

\begin{lstlisting}[style=tex]
\newenvironment{Magic}[1][Pink]% hat 1 optionales Arg., Standardwert: Pink
   {\begin{center}\begingroup\textcolor{#1}}%
   {\endgroup\end{center}}\end{lstlisting}
\pause

\begin{multicols}{2}
Hier ein Beispiel:
\begin{lstlisting}[style=tex]
\begin{Magic}
   mein toller Text
\end{Magic}
\end{lstlisting}

Und so sieht es aus:
\vspace{0.75ex}
\begin{Magic}
   mein toller Text
\end{Magic}
\vspace{0.75ex}
\end{multicols}
\end{frame}
%\input{slides/tikz}
%\section{Bibliographie-Verzeichnis}
\begin{frame}[fragile]{Bibliographie-Verzeichnis mit Bib\LaTeX}
Wenn man sich auf Bücher oder Artikel anderer Autoren bezieht, muss man dies kennzeichnen. Dazu erstellt man ein Bibliographie-Datenbank und bindet diese ein:
\begin{lstlisting}[style=tex]
\usepackage[OPTIONS]{biblatex} % z.B. style=numeric

\addbibresource{LITERATURDATENBANK.bib} % bindet die Literaturdatenbank ein

\begin{document}
   \printbibliography % erstellt das Literaturverzeichnis
\end{document}
\end{lstlisting}
\end{frame}

\begin{frame}[fragile]{Bibliographie-Verzeichnis mit Bib\LaTeX}
Ein Eintrag in der Datenbank hat folgendes Schema:
\begin{lstlisting}[style=tex]
@REFERENZART{key,
author = "",
title = "",
...
}
\end{lstlisting}

Es gibt verschiedene Referenzarten:
\begin{small}
	\vspace{-1ex}
	\begin{multicols}{3}
		\begin{itemize}\setlength{\itemsep}{-0.5ex}
			\item article
			\item book
			\item mvbook
			\item inbook
			\item bookinbook
			\item suppbook
			\item booklet
			\item collection
			\item mvcollection
			\item incollection
			\item suppcollection
			\item manual
			\item misc
			\item online
			\item patent
			\item periodical
			\item suppperiodical
			\item proceedings
			\item mvproceedings
			\item inproceedings
			\item reference
			\item mvreference
			\item inreference
			\item report
			\item set
			\item thesis
			\item unpublished
			\item xdata
			\item custom[a-f]
		\end{itemize}
	\end{multicols}
\end{small}

Quelle: \url{http://mirror.utexas.edu/ctan/info/translations/biblatex/de/biblatex-de-Benutzerhandbuch.pdf}
\end{frame}

\begin{frame}[fragile]{Bibliographie-Verzeichnis mit Bib\LaTeX}
Das Paket \textbf{\texttt{biblatex}} unterstützt sowohl die nummerische Zitierweise, als auch das Autor-Jahr-Schema. Einige der unterschidlichen Zitierstile sind in der nachfolgenden Tabelle zusammengefasst. 
\begin{table}
	\caption{Einige Zitierstile von Bib\LaTeX}
\begin{tabular}{ll}
	\hline
	Zitierstil & Beispiel-Ausgabe \\
	\hline
	\texttt{numeric} & [1, 4, 3, 5] \\
	\texttt{numeric-comp} & [1, 3-5] \\
	\texttt{numeric-verb} & [1]; [4]; [3]; [5] \\
	\texttt{alphabetic} & [Ang02; Spr50; GMS94; VLUR93] \\
	\texttt{alphabetic-verb} & [Ang02]; [Spr50]; [GMS94]; [VLUR93] \\
	\texttt{authoryear} & Angenendt 2002 \\
	\texttt{authortitle} & Angenendt, >>In Honore Salvatoris<< \\
	\hline
\end{tabular}
\end{table}

Der gewünschte Stil wird als Paketoption an \texttt{biblatex} übergeben, also bspw. \texttt{style=authoryear}:
\begin{lstlisting}[style=tex]
\usepackage[style=authoryear, backend=biber]{biblatex}
\end{lstlisting}
\end{frame}

\begin{frame}[fragile]{Bibliographie-Verzeichnis mit Bib\LaTeX}
\begin{table}
	\caption{Zusammenstellung der Pakete, die zusätzliche Stile für \texttt{biblatex} bereitstellen}
	\begin{tabular}{ll}
		\hline
		Stil & Hinweis \\
		\hline
		\texttt{apa} & American Psychological Association \\
		\texttt{authoryear-icomp-tt} & \texttt{authoryear-comp} und \texttt{authoryear-ibid} \\
		\texttt{chem-angew} & Angewandte Chemie \\
		\texttt{chem-biochem} & Biochemistry \\
		\texttt{chicago} & Chicago Manual Style \\
		\texttt{footnote-dw} & Vollzitat als Fußnote \\
		\texttt{authortitle-dw} & Autor-Titel-Stil \\
		\texttt{historian} & Standard \\
		\texttt{biblatex-jura} & Juristische Schriften \\
		\texttt{mla} & Modern Language Association \\
		\texttt{nature} & Nature Journal \\
		\texttt{science} & Science Journal \\
		\texttt{historische-zeitschrift} & Historische Zeitschrift \\
		\texttt{philosophy-classic} & \\
		\hline
	\end{tabular}
\end{table}

\end{frame}

\begin{frame}[fragile]{Bibliographie-Verzeichnis mit Bib\LaTeX}
Standardmakros für Verweise:
\begin{lstlisting}[sytle=tex, numbers=none]
\cite[Präfix][Suffix]{Schlüssel}<Satzzeichen>
\end{lstlisting}
\begin{lstlisting}[style=tex, numbers=none]
\Cite[Präfix][Suffix]{Schlüssel}<Satzzeichen>
\end{lstlisting}
\begin{lstlisting}[style=tex, numbers=none]
\parencite[Präfix][Suffix]{Schlüssel}<Satzzeichen>
\end{lstlisting}
\begin{lstlisting}[style=tex, numbers=none]
\Parencitecite[Präfix][Suffix]{Schlüssel}<Satzzeichen>
\end{lstlisting}
\begin{lstlisting}[style=tex, numbers=none]
\footcite[Präfix][Suffix]{Schlüssel}<Satzzeichen>
\end{lstlisting}
\begin{lstlisting}[style=tex, numbers=none]
\footcitetext[Präfix][Suffix]{Schlüssel}<Satzzeichen>
\end{lstlisting}
\end{frame}

\begin{frame}[fragile]{Bibliographie-Verzeichnis mit Bib\LaTeX}
\begin{lstlisting}[style=tex]
\parencite[121]{springer}
\end{lstlisting}
\begin{table}
	\caption{Das Makro \texttt{\textbackslash parencite}}
\begin{tabular}{ll}
	\hline
	Zitierstil & Beispiel-Ausgabe \\
	\hline
	\texttt{numeric} & [1, S. 121] \\
	\texttt{alphabetic} & [Spr50, S. 121] \\
	\texttt{authoryear} & (Springer 1950, S. 121) \\
	\texttt{authortitle} & (Springer, >>Mediaeval Pilgrim Routes<<, S. 121) \\
	%\texttt{verbose} & (Springer, >>Mediaeval Pilgrim Routes<<, S. 121) \\
	\hline
\end{tabular}
\begin{lstlisting}[style=tex]
\textcite[121]{springer}
\end{lstlisting}
\begin{table}
	\caption{Das Makro \texttt{\textbackslash textcite}}
	\begin{tabular}{ll}
		\hline
		Zitierstil & Beispiel-Ausgabe \\
		\hline
		\texttt{numeric} & Springer [1, S. 121] \\
		\texttt{alphabetic} & Springer [Spr50, S 121] \\
		\texttt{authoryear} & Springer [1950, S. 121] \\
		\texttt{authortitle} & Springer (>>Mediaeval Pilgrim Routes<<, S. 121) \\
		\hline
	\end{tabular}
\end{table}
\end{table}

\end{frame}


%\begin{frame}[fragile]{Bibliographie-Verzeichnis mit Bib\LaTeX}
%Ein Eintrag in der Datenbank hat folgendes Schema:
%\begin{lstlisting}[style=tex]
%@REFERENZART{NAME,
%   author = "",
%   title = "",
%   volume = "",
%   number = "",
%   pages = "",
%   year = "", % hat die Form JJJJ
%   month = "",
%   howpublished = "",
%   note = "",
%   url = "", % benötigt das Paket "hyperref"
%}
%\end{lstlisting}

%Es gibt verschiedene Referenzarten:
%\begin{small}
%\vspace{-1ex}
%\begin{multicols}{3}
%\begin{itemize}\setlength{\itemsep}{-0.5ex}
%\item article
%\item book
%\item booklet
%\item conference
%\item inbook
%\item incollection
%\item inproceedings
%\item manual
%\item masterthesis
%\item misc
%\item phdthesis
%\item proceedings
%\item tecreport
%\item unpublished
%\end{itemize}
%\end{multicols}
%\end{small}

%Quelle: \url{https://de.wikipedia.org/wiki/BibTeX}
%\end{frame}



%\begin{frame}[fragile]{Bibliographie-Verzeichnis mit Bib\LaTeX}
%Eine Ressource zitiert man im Text wie folgt:
%\begin{lstlisting}[style=tex]
%\cite{NAME}
%\end{lstlisting}
%Sie wird dann im Literaturverzeichnis aufgeführt.%\pause

%\bigskip
%Sollen alle Einträge der Literaturdatenbank in das Literaturverzeichnis übernommen
%werden, wenngleich sie nicht im Text zitiert wurden, kann dies wie folgt erreicht werden:
%\begin{lstlisting}[style=tex]
%\nocite{*}
%\end{lstlisting}
%\end{frame}
%\input{slides/ende}
\end{document}
