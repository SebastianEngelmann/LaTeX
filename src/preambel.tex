% //////////////////// Pakete laden ////////////////////
\usepackage{amsmath}			% MUSS vor fontspec geladen werden
\usepackage{mathtools}			% modifiziert amsmath
\usepackage{amssymb}			% mathematische symbole, für \ceckmarks
\usepackage{amsthm}				% für proof
\usepackage{mathrsfs}			% für \mathscr
\usepackage{latexsym}

\usepackage{fontspec} 			% funktioniert nur mit den neueren Compilern z.B. XeLaTeX
%\usepackage[T1]{fontenc}
%\usepackage[utf8]{inputenc}		% funktioniert mit PDFLaTeX
\usepackage{microtype}			% für bessere Worttrennung
\usepackage[ngerman]{babel} 	% Spracheinstellung
\usepackage{lmodern}			% verändert verwendete Schriftart, damit sie weniger pixelig ist

\usepackage{verbatim}
\usepackage{listings}			% Für Quellcode
\usepackage{tabularx}
\usepackage{csquotes}

\usepackage{graphicx}
\usepackage{multirow}			% für multirow in tabulars

\PassOptionsToPackage{cmyk,table}{xcolor} % weil xcolor von anderem paket bereits geladen wird
%\usepackage[cmyk,table]{xcolor} % um Farben zu benutzen, kann mehr als das Paket color
\PassOptionsToPackage{colorlinks,linktocpage,linkcolor=blue}{hyperref}
% \usepackage[					% Verlinkungen
% 	colorlinks,					% farbige Schrift, statt farbiger Rahmen
% 	linktocpage,				% verlinkt im Abb.Verzeichnis Seitenzahl statt Bildunterschrift
% 	linkcolor=blue				% setzt Farbe der Links auf blau
% 	]{hyperref}					% nur für digitale Anwendungen, url = "http://www.example.com"
\usepackage{url}				% für Webadressen wie e-mail usw.: "\url{http://www.example.com}"
\renewcommand\UrlFont{\footnotesize\color{FUblue}\rmfamily}

\usepackage{enumerate}			% für versch. Aufzählungezeichen wie z.B. a)
\usepackage{float}
\usepackage{xspace}				% folgt ein Leerzeichen nach einem \Befehl, wird es nicht verschluckt.
\usepackage{multicol}			% für mehrspaltiges Layout

\usepackage{fp}
\usepackage{tikz}
\usetikzlibrary{tikzmark}			% für \tikzmark{toRemember}
\usetikzlibrary{positioning}	% verbesserte Positionierung der Knoten
\usetikzlibrary{automata}		% für Automaten (GTI)
\usetikzlibrary{arrows}
\usetikzlibrary{shapes}
\usetikzlibrary{decorations.pathmorphing}
\usetikzlibrary{decorations.pathreplacing}
\usetikzlibrary{decorations.shapes}
\usetikzlibrary{decorations.text}

\usepackage{pstricks}
\usepackage{pst-node}
\usepackage{pst-grad}

\usepackage{texnames}

\usepackage{xltxtra}

% //////////////////// eigene Farben ////////////////////
\let\definecolor=\xdefinecolor
\definecolor{FUgreen}{RGB}{153,204,0}
\definecolor{FUblue}{RGB}{0,51,102}

\definecolor{middlegray}{rgb}{0.5,0.5,0.5}
\definecolor{lightgray}{rgb}{0.8,0.8,0.8}
\definecolor{orange}{rgb}{0.8,0.3,0.3}
\definecolor{azur}{rgb}{0,0.7,1}
\definecolor{yac}{rgb}{0.6,0.6,0.1}
\definecolor{Pink}{rgb}{1,0,0.6}

\definecolor{bgcolour}{rgb}{0.97,0.97,0.97}
\definecolor{codegreen}{rgb}{0,0.6,0}
\definecolor{codegray}{rgb}{0.35,0.35,0.35}
\definecolor{codepurple}{rgb}{0.58,0,0.82}
\definecolor{codeblue}{rgb}{0.4,0.5,1}

% //////////////////// Syntaxhighlighting ////////////////////
\lstloadlanguages{Python, Haskell, [LaTeX]TeX, Java}
\lstset{
   basicstyle=\small\fontfamily{fvm}\selecfont,	% \scriptsize the size of the fonts that are used for the code
   backgroundcolor = \color{bgcolour},	% legt Farbe der Box fest
   breakatwhitespace=false,	% sets if automatic breaks should only happen at whitespace
   %breaklines=true,			% sets automatic line breaking
   captionpos=t,				% sets the caption-position to bottom, t for top
   commentstyle=\color{codeblue}\ttfamily,% comment style
   frame=single,				% adds a frame around the code
   keepspaces=true,			% keeps spaces in text, useful for keeping indentation
							% of code (possibly needs columns=flexible)
   keywordstyle=\bfseries\color{blue},% keyword style
   numbers=left,				% where to put the line-numbers;
   							% possible values are (none, left, right)
   numberstyle=\tiny\color{codegreen},	% the style that is used for the line-numbers
   numbersep=5pt,			% how far the line-numbers are from the code
   stepnumber=1,				% nummeriert nur jede i-te Zeile
   showspaces=false,			% show spaces everywhere adding particular underscores;
							% it overrides 'showstringspaces'
   %showstringspaces=false,	% underline spaces within strings only
   showtabs=false,			% show tabs within strings adding particular underscores
   flexiblecolumns=false,
   %tabsize=1,				% the step between two line-numbers. If 1: each line will be numbered
   stringstyle=\color{orange}\ttfamily,	% string literal style
   numberblanklines=false,				% leere Zeilen werden nicht mitnummeriert
   xleftmargin=1.2em,					% Abstand zum linken Layoutrand
   xrightmargin=0.4em,					% Abstand zum rechten Layoutrand
   aboveskip=2ex, 
}

\lstdefinestyle{py}{
   language=Python,
}
\lstdefinestyle{hs}{
   language=Haskell,
}
\lstdefinestyle{tex}{
	language=[LaTeX]TeX,
	escapeinside={\%*}{*)},     % if you want to add LaTeX within your code
	texcsstyle=*\bfseries\color{blue},% hervorhebung der tex-Schlüsselwörter
	morekeywords={*,$,\{,\},\[,\],lstinputlisting,includegraphics,listoffigures,lstlistoflistings,subsection,subsubsection,textcolor,tableofcontents,colorbox,fcolorbox,definecolor,cellcolor,url,linktocpage,subtitle,subject,maketitle,usetikzlibrary,node,path,addbibresource,printbibliography},% if you want to add more keywords to the set
}
\lstdefinestyle{java}{
	language=Java,
	extendedchars=true,		% lets you use non-ASCII characters;
   						% for 8-bits encodings only, does not work with UTF-8
}

% //////////////////// eigene Kommandos ////////////////////
\newcommand{\Quellcode}[3]{\lstinputlisting[style=#2, caption={#3}]{#1.#2}}% 1. path/filename, 2. type, 3. beschr.
\newcommand\zz{\ensuremath{\raisebox{+0.25ex}{Z}% zu zeigen
			\kern-0.4em\raisebox{-0.25ex}{Z}%
			\;\xspace}%
}
\newenvironment{Magic}[1][Pink]% hat 1 optionales Arg., Standardwert: Pink
   {\begin{center}\begingroup\color{#1}\huge}%
   {\endgroup\end{center}}
