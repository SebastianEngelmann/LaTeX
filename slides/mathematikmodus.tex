\section{Mathematikmodus}

% ===== Mathematikmodus =====
\begin{frame}[fragile]{Mathematikmodus}
Der Mathematikmodus ist eine Umgebung, die dafür optimiert ist, mathematische Formeln und Symbole darzustellen. Dafür werden folgende Pakete benötigt:
%Ist eine Umgebung, die dafür optimiert ist, mathematische Formeln und Symbole darzustellen.
%Benötigte Pakete:
\begin{lstlisting}[style=tex]
\usepackage{amsmath}    % MUSS vor fontspec geladen werden
\usepackage{mathtools}   % modifiziert amsmath
\usepackage{amssymb}    % mathematische symbole, für \checkmarks
\usepackage{amsthm}      % für proof
\usepackage{mathrsfs}     % für \mathscr
\usepackage{latexsym}
\usepackage{marvosym}   % zusätzliche Zeichen, z.B. Lightning
\usepackage{cancel}         % für das Durchstreichen u.a. in Matheformeln mit \cancel
\end{lstlisting}
\end{frame}

% \begin{frame}[fragile]{Mathematikmodus}
% \begin{itemize}
% \item zahlreiche Befehle, Symbole und Umgebungen funktionieren nur in diesem Modus
% \item andere funktionieren gerade in diesem Modus nicht und man muss für diese entsprechend andere Modi oder Befehle verwenden:

% \vspace{1ex}
% \begin{tabular}{l|l|l}
% Befehl & Textmodus & Mathematikmodus \\
% \hline
% Unterstrich & \verb|\_| & \_ \\
% Dach & \verb|\^{}| & \verb|^| \\
% Tabelle & \verb|tabular| & \verb|array| \\
% Bold & \verb|\textbf{}| & \verb|\mathbf{}| \\
% Backslash & \verb|\textbackslash| & \verb|\backslash| \\
% \end{tabular}
% \end{itemize}
% \end{frame}

\subsection{inline vs. abgesetzt}
\begin{frame}[fragile]{inline vs. abgesetzt}
\begin{itemize}
\item Im Fließtext wird der betreffende Ausdruck mit Dollarzeichen \textbf{\$} umgeben. z.B.: \verb|$E = mc^2$|
% \begin{lstlisting}[style=tex]
% Mitten im $E = mc^2$ Text...
% \end{lstlisting}

\vspace{0.5ex} Mitten im Text steht die Formel $E = mc^2$ von Einstein.\pause

\item Um einzeilige Formeln abzusetzen, baut man eine Umgebung auf, innerhalb derer der Mathematikmodus eingeschaltet ist:

\begin{lstlisting}[style=tex]
\[ E=mc^2 \]
\end{lstlisting}
Führt zu:
\[ E=mc^2 \]
\end{itemize}
\end{frame}

\begin{frame}[fragile]{inline vs. abgesetzt}
\begin{itemize}
\item Die \textbf{align}-Umgebung schaltet den Mathematikmodus ein,
\item zentriert Formel,
\item erlaubt Zeilenumbrüche (\textbf{\textbackslash\textbackslash}),
\item nummeriert die Zeilen und
\item ermöglicht Ausrichtung der Zeilen zueinander (mittels \textbf{\&}).
\end{itemize}

\begin{lstlisting}[style=tex]
\begin{align}
\sum_{i=0}^{n+1} i &= \sum_{i=0}^n i + (n+1) \\
                                    &= \frac{n(n+1)}{2} + (n+1)
\end{align}
\end{lstlisting}

\pause Und so sieht es aus:
\begin{align}
\sum_{i=0}^{n+1} i &= \sum_{i=0}^n i + (n+1) \\
                       &= \frac{n(n+1)}{2} + (n+1)
\end{align}\pause

Will man keine Nummerierung, nutzt man \textbf{align*}
\end{frame}

\subsection{Spezielle Formatierungen}
\begin{frame}[fragile]{Spezielle Formatierungen}
\begin{minipage}{0.3\linewidth}
\begin{lstlisting}[style=tex]
\begin{theorem}
...
\end{theorem}
\end{lstlisting}
\end{minipage}
\hfill
\begin{minipage}{0.65\linewidth}
\begin{theorem}
This is a theorema about right triangles and can be summarised in the next 
equation 
\[ x^2 + y^2 = z^2 \]
\end{theorem}
\end{minipage}

\begin{minipage}{0.3\linewidth}
\begin{lstlisting}[style=tex]
\begin{corollary}
...
\end{corollary}
\end{lstlisting}
\end{minipage}
\hfill
\begin{minipage}{0.65\linewidth}
\begin{corollary}
There's no right rectangle whose sides measure 3cm, 4cm, and 6cm.
\end{corollary}
\end{minipage}

\begin{minipage}{0.3\linewidth}
\begin{lstlisting}[style=tex]
\begin{lemma}
...
\end{lemma}
\end{lstlisting}
\end{minipage}
\hfill
\begin{minipage}{0.65\linewidth}
\begin{lemma}
Given two line segments whose lengths are $a$ and $b$ respectively there is a 
real number $r$ such that $b=r\cdot a$.
\end{lemma}
\end{minipage}
\end{frame}

\begin{frame}[fragile]{Spezielle Formatierungen}
\begin{minipage}{0.3\linewidth}
\begin{lstlisting}[style=tex]
\begin{proof}
...
\end{proof}
\end{lstlisting}
\end{minipage}
\hfill
\begin{minipage}{0.65\linewidth}
\begin{proof}
Mit $r=\frac{b}{a}$ gilt stets $b=r\cdot a$.
\end{proof}
\end{minipage}

\begin{minipage}{0.3\linewidth}
\begin{lstlisting}[style=tex]
\begin{definition}
...
\end{definition}
\end{lstlisting}
\end{minipage}
\hfill
\begin{minipage}{0.65\linewidth}
\begin{definition}
\[ fak(n) := \begin{cases}
1, & n = 0 \\
n\cdot fak(n-1), & \text{sonst}
\end{cases}
\quad
\forall n \in \mathbb{N} \]
\end{definition}
\end{minipage}
\end{frame}