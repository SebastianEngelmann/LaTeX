\section{Aufbau eines Dokuments}

% ////////////////// Dokumentenklassen //////////////
\subsection{Dokumententypen}
\begin{frame}[fragile]{Dokumentenklassen}
\begin{itemize}
\item \LaTeX{} bringt standardmäßig verschiedene Grunddokumententypen mit, die \textbf{verschiedene Eigenschaften} haben \newline
\item in jedem Dokument muss die \textbf{Dokumentenklasse} angegeben werden $\Rightarrow$ erster Befehl eines Dokuments \newline
\item Befehl zur Angabe der Klasse beinhaltet die Dokumentenklasse (in geschweiften Klammern) sowie die Angabe möglicher \textbf{Optionen} (in eckigen Klammern)
\end{itemize}
%\begin{varwidth}{0.45\linewidth}
\begin{lstlisting}[style=tex]
\documentclass[Optionen]{Dokumentenklasse}\end{lstlisting}
\begin{itemize}
\item \textbf{Übersicht:} \url{http://www.kkittel.de/wiki/doku.php?id=grundlegende_einstellungen:dokumentenklassen}
\end{itemize}
\end{frame}


% ////////////////// Typen Übersicht //////////////
\begin{frame}{Dokumententypen}
\begin{itemize}
\item \textbf{Article:} geeignet für kurze technische Artikel \textit{(scrartcl)}
\item \textbf{Report:} geeignet für längere technische Artikel \textit{(scrreprt)}
\item \textbf{Book:} Drucklayout standardmäßig zweiseitig; z.B.: automatische Kopfzeile mit Seitenzahl \textit{(scrbook)}

\item \textbf{Letter:} geeignet zum Schreiben von Briefen \textit{(scrlttr2)}
\end{itemize}
\end{frame}

% ///////////// Pakete ///////////////
\subsection{Pakete}
\begin{frame}[fragile]{Pakete}
\TeX{} und \LaTeX{} besitzen einen Grundstock an Befehlen. Wenn man darüber hinaus Modifizierungen vornehmen möchte, muss man zusätzliche Pakete einbinden, die weitere Befehle zur Verfügung stellen.\pause

\begin{lstlisting}[style=tex]
\usepackage{fontspec}
    % Schriftpaket, funktioniert nur mit den neueren Compilern z.B. XeLaTeX
\usepackage{microtype}
    % verbessert die Worttrennung
\usepackage[ngerman]{babel}
    % Spracheinstellung: richtige Silbentrennung.
\usepackage{lmodern}
    % verändert verwendete Schriftart, damit sie weniger pixelig ist
\end{lstlisting}
\end{frame}

% ////////////// Begin Dokument //////////////
\subsection{Das Dokument}
\begin{frame}[fragile]{Das Dokument}
Im Kopf des Dokuments kann man Einstellungen vornehmen und zusätzliche Pakete einbinden. Der eigentliche Inhalt folgt dann in der \textbf{document}-Umgebung:
\begin{lstlisting}[style=tex]
\documentclass[paper=a4, 10pt, ngerman]{scrartcl}

\begin{document}
   Mein Inhalt
\end{document}\end{lstlisting}
\end{frame}

% //////// Titlepage //////////////
\subsection{Titelseite}
\begin{frame}[fragile]{Titelseite}
Die für ein Dokument wichtige Angaben (wie z.B. Titel, Autor und Datum) lassen sich sehr einfach erstellen und werden dann automatisch vorformatiert.
\begin{lstlisting}[style=tex]
\begin{titlepage}
  \title{Crashkurs LaTeX}
  \subtitle{Mentoring WiSe 2016}
  \subject{Computer Science}
  \author{Anja Wolffgramm, Diane Hanke}
  \date{25. November 2016}
\end{titlepage}

\maketitle % erstellt die Titelseite
\end{lstlisting}

% \textbf{Anwendung:}
% \vspace{-2ex}
% \begin{lstlisting}[style=tex]
% \begin{titlepage}
%   ...
% \end{titlepage}

% \begin{document}
%    \maketitle
% \end{document}
% \end{lstlisting}
% Wendet man die Zeilen 1 -- 7 vor Beginn des Dokuments an, wird eine extra Cover-Seite erstellt.

% \begin{lstlisting}[style=tex]
% \begin{document}
%   \begin{titlepage}
%     ...
%   \end{titlepage}
  
%   \maketitle
% \end{document}
% \end{lstlisting}
% Wendet man es innerhalb des Dokuments an, wird nachfolgender Inhalt auf die selbe Seite gesetzt.
\end{frame}
