\section{Funktionsweise}

\begin{frame}{Funktionsweise}
Für einfache Texte gelten folgende Schritte:
\begin{enumerate}
	\item Texte im Editor schreiben, bspw. mit \textit{Kile}: \texttt{kile meinText.tex}
	\item Dokument übersetzen, bspw. mit pdf\LaTeX{}: \texttt{pdflatex meinText.tex}
	\newline
\end{enumerate}
Es werden dabei mindestens folgende Dateien erzeugt:
\begin{tabular}[pos]{lp{8.5cm}}
	\texttt{meinText.log} & Enthält alle Statusmeldungen des Übersetzungsvorgangs. \\
	\texttt{meinText.aux} & Enthält unter anderem die Einträge für Querverweise. \\
	\texttt{meinText.pdf} & Das erzeugte PDF-Dokument (nach dem ersten Durchlauf ohne Inhaltsverzeichnis).
\end{tabular}
\begin{enumerate}
	\item[3.] Erneutes Übersetzen des Dokuments: \texttt{pdflatex meinText.tex}
	\item[4.] PDF-Dokument ansehen, bspw. mit dem PDF-Viewer \textit{okular}: \texttt{okular meinText.pdf}
\end{enumerate}
\end{frame}

\begin{frame}{Funktionsweise}
\begin{itemize}
	\item Je nach Komplexität besteht die Notwendigkeit mehrerer Durchläufe des \TeX{}-Compilers
	\newline
	\item Beim ersten Durchlauf erzeugt der \TeX{}-Compiler aus allen Überschriften das Inhaltsverzeichnis (Die Informationen werden in einer Hilfdatei gespeichert -- Dataiendung \texttt{.aux})
	\newline
	\item Da das Inhaltsverzeichnis am Anfang des Dokuments steht, liegt es beim ersten Start noch nicht vor
	\newline
	\item Erst beim zweiten Durchlauf wird das Inhaltsverzeichnis eingebunden
	\newline
	\item Dies kann aber dazu führen, dass alle nachfolgenden Seiten nach hinten verschoben werden, wenn das Inhaltsverzeichnis mehrere Seiten umfasst und so eventuell Verweise auf Seitenzahlen nicht mehr stimmen
	\newline
	\item Somit muss ein dritter Durchlauf gestartet werden
\end{itemize}
\end{frame}
