\section{Bibliographie-Verzeichnis}
\begin{frame}[fragile]{Bibliographie-Verzeichnis mit Bib\LaTeX}
Wenn man sich auf Bücher oder Artikel anderer Autoren bezieht, muss man dies kennzeichnen. Dazu erstellt man ein Bibliographie-Datenbank und bindet diese ein:
\begin{lstlisting}[style=tex]
\usepackage[OPTIONS]{biblatex} % z.B. style=numeric

\addbibresource{LITERATURDATENBANK.bib} % bindet die Literaturdatenbank ein

\begin{document}
   \printbibliography % erstellt das Literaturverzeichnis
\end{document}
\end{lstlisting}
\end{frame}

\begin{frame}[fragile]{Bibliographie-Verzeichnis mit Bib\LaTeX}
Ein Eintrag in der Datenbank hat folgendes Schema:
\begin{lstlisting}[style=tex]
@REFERENZART{key,
author = "",
title = "",
...
}
\end{lstlisting}

Es gibt verschiedene Referenzarten:
\begin{small}
	\vspace{-1ex}
	\begin{multicols}{3}
		\begin{itemize}\setlength{\itemsep}{-0.5ex}
			\item article
			\item book
			\item mvbook
			\item inbook
			\item bookinbook
			\item suppbook
			\item booklet
			\item collection
			\item mvcollection
			\item incollection
			\item suppcollection
			\item manual
			\item misc
			\item online
			\item patent
			\item periodical
			\item suppperiodical
			\item proceedings
			\item mvproceedings
			\item inproceedings
			\item reference
			\item mvreference
			\item inreference
			\item report
			\item set
			\item thesis
			\item unpublished
			\item xdata
			\item custom[a-f]
		\end{itemize}
	\end{multicols}
\end{small}

Quelle: \url{http://mirror.utexas.edu/ctan/info/translations/biblatex/de/biblatex-de-Benutzerhandbuch.pdf}
\end{frame}

\begin{frame}[fragile]{Bibliographie-Verzeichnis mit Bib\LaTeX}
Das Paket \texttt{biblatex} unterstützt sowohl die nummerische Zitierweise, als auch das Autor-Jahr-Schema. Einige der unterschidlichen Zitierstile sind in der nachfolgenden Tabelle zusammengefasst. 
\begin{table}
	\caption{Einige Zitierstile von Bib\LaTeX}
\begin{tabular}{ll}
	\hline
	Zitierstil & Beispiel-Ausgabe \\
	\hline
	\texttt{numeric} & [1, 4, 3, 5] \\
	\texttt{numeric-comp} & [1, 3-5] \\
	\texttt{numeric-verb} & [1]; [4]; [3]; [5] \\
	\texttt{alphabetic} & [Ang02; Spr50; GMS94; VLUR93] \\
	\texttt{alphabetic-verb} & [Ang02]; [Spr50]; [GMS94]; [VLUR93] \\
	\texttt{authoryear} & Angenendt 2002 \\
	\texttt{authortitle} & Angenendt, >>In Honore Salvatoris<< \\
	\hline
\end{tabular}
\end{table}

Der gewünschte Stil wird als Paketoption an \texttt{biblatex} übergeben, also bspw. \texttt{style=authoryear}:
\begin{lstlisting}[style=tex]
\usepackage[style=authoryear, backend=biber]{biblatex}
\end{lstlisting}
\end{frame}

\begin{frame}[fragile]{Bibliographie-Verzeichnis mit Bib\LaTeX}
\begin{table}
	\caption{Zusammenstellung der Pakete, die zusätzliche Stile für \texttt{biblatex} bereitstellen}
	\begin{tabular}{ll}
		\hline
		Stil & Hinweis \\
		\hline
		\texttt{apa} & American Psychological Association \\
		\texttt{authoryear-icomp-tt} & \texttt{authoryear-comp} und \texttt{authoryear-ibid} \\
		\texttt{chem-angew} & Angewandte Chemie \\
		\texttt{chem-biochem} & Biochemistry \\
		\texttt{chicago} & Chicago Manual Style \\
		\texttt{footnote-dw} & Vollzitat als Fußnote \\
		\texttt{authortitle-dw} & Autor-Titel-Stil \\
		\texttt{historian} & Standard \\
		\texttt{biblatex-jura} & Juristische Schriften \\
		\texttt{mla} & Modern Language Association \\
		\texttt{nature} & Nature Journal \\
		\texttt{science} & Science Journal \\
		\texttt{historische-zeitschrift} & Historische Zeitschrift \\
		\texttt{philosophy-classic} & \\
		\hline
	\end{tabular}
\end{table}

\end{frame}

\begin{frame}[fragile]{Bibliographie-Verzeichnis mit Bib\LaTeX}
Standardmakros für Verweise:
\begin{lstlisting}[sytle=tex]
\cite[Präfix][Suffix]{Schlüssel}<Satzzeichen>
\end{lstlisting}
\begin{lstlisting}[style=tex]
\Cite[Präfix][Suffix]{Schlüssel}<Satzzeichen>
\end{lstlisting}
\begin{lstlisting}[style=tex]
\parencite[Präfix][Suffix]{Schlüssel}<Satzzeichen>
\end{lstlisting}
\begin{lstlisting}[style=tex]
\Parencitecite[Präfix][Suffix]{Schlüssel}<Satzzeichen>
\end{lstlisting}
\begin{lstlisting}[style=tex]
\footcite[Präfix][Suffix]{Schlüssel}<Satzzeichen>
\end{lstlisting}
\begin{lstlisting}[style=tex]
\footcitetext[Präfix][Suffix]{Schlüssel}<Satzzeichen>
\end{lstlisting}
\end{frame}

\begin{frame}[fragile]{Bibliographie-Verzeichnis mit Bib\LaTeX}
\begin{lstlisting}[style=tex]
\parencite[121]{springer}
\end{lstlisting}
\begin{table}
	\caption{Das Makro \texttt{\textbackslash parencite}}
\begin{tabular}{ll}
	\hline
	Zitierstil & Beispiel-Ausgabe \\
	\hline
	\texttt{numeric} & [1, S. 121] \\
	\texttt{alphabetic} & [Spr50, S. 121] \\
	\texttt{authoryear} & (Springer 1950, S. 121) \\
	\texttt{authortitle} & (Springer, >>Mediaeval Pilgrim Routes<<, S. 121) \\
	%\texttt{verbose} & (Springer, >>Mediaeval Pilgrim Routes<<, S. 121) \\
	\hline
\end{tabular}
\begin{lstlisting}[style=tex]
\textcite[121]{springer}
\end{lstlisting}
\begin{table}
	\caption{Das Makro \texttt{\textbackslash textcite}}
	\begin{tabular}{ll}
		\hline
		Zitierstil & Beispiel-Ausgabe \\
		\hline
		\texttt{numeric} & Springer [1, S. 121] \\
		\texttt{alphabetic} & Springer [Spr50, S 121] \\
		\texttt{authoryear} & Springer [1950, S. 121] \\
		\texttt{authortitle} & Springer (>>Mediaeval Pilgrim Routes<<, S. 121) \\
		\hline
	\end{tabular}
\end{table}
\end{table}

\end{frame}


%\begin{frame}[fragile]{Bibliographie-Verzeichnis mit Bib\LaTeX}
%Ein Eintrag in der Datenbank hat folgendes Schema:
%\begin{lstlisting}[style=tex]
%@REFERENZART{NAME,
%   author = "",
%   title = "",
%   volume = "",
%   number = "",
%   pages = "",
%   year = "", % hat die Form JJJJ
%   month = "",
%   howpublished = "",
%   note = "",
%   url = "", % benötigt das Paket "hyperref"
%}
%\end{lstlisting}

%Es gibt verschiedene Referenzarten:
%\begin{small}
%\vspace{-1ex}
%\begin{multicols}{3}
%\begin{itemize}\setlength{\itemsep}{-0.5ex}
%\item article
%\item book
%\item booklet
%\item conference
%\item inbook
%\item incollection
%\item inproceedings
%\item manual
%\item masterthesis
%\item misc
%\item phdthesis
%\item proceedings
%\item tecreport
%\item unpublished
%\end{itemize}
%\end{multicols}
%\end{small}

%Quelle: \url{https://de.wikipedia.org/wiki/BibTeX}
%\end{frame}



%\begin{frame}[fragile]{Bibliographie-Verzeichnis mit Bib\LaTeX}
%Eine Ressource zitiert man im Text wie folgt:
%\begin{lstlisting}[style=tex]
%\cite{NAME}
%\end{lstlisting}
%Sie wird dann im Literaturverzeichnis aufgeführt.%\pause

%\bigskip
%Sollen alle Einträge der Literaturdatenbank in das Literaturverzeichnis übernommen
%werden, wenngleich sie nicht im Text zitiert wurden, kann dies wie folgt erreicht werden:
%\begin{lstlisting}[style=tex]
%\nocite{*}
%\end{lstlisting}
%\end{frame}