\section{Bibliographie-Verzeichnis}
\begin{frame}[fragile]{Bibliographie-Verzeichnis mit Bib\TeX}
Wenn man sich auf Bücher oder Artikel anderer Autoren bezieht, muss man dies kennzeichnen. Dazu erstellt man ein Bibliographie-Datenbank und bindet diese ein:
\begin{lstlisting}[style=tex]
\usepackage[OPTIONS]{biblatex} % z.B. style=numeric

\addbibresource{LITERATURDATENBANK.bib} % bindet die Literaturdatenbank ein

\begin{document}
   \printbibliography % erstellt das Literaturverzeichnis
\end{document}
\end{lstlisting}
\end{frame}

\begin{frame}[fragile]{Bibliographie-Verzeichnis mit Bib\TeX}
Ein Eintrag in der Datenbank hat folgendes Schema:
\begin{lstlisting}[style=tex]
@REFERENZART{NAME,
   author = "",
   title = "",
   volume = "",
   number = "",
   pages = "",
   year = "", % hat die Form JJJJ
   month = "",
   howpublished = "",
   note = "",
   url = "", % benötigt das Paket "hyperref"
}
\end{lstlisting}

Es gibt verschiedene Referenzarten:
\begin{small}
\vspace{-1ex}
\begin{multicols}{3}
\begin{itemize}\setlength{\itemsep}{-0.5ex}
\item article
\item book
\item booklet
\item conference
\item inbook
\item incollection
\item inproceedings
\item manual
\item masterthesis
\item misc
\item phdthesis
\item proceedings
\item tecreport
\item unpublished
\end{itemize}
\end{multicols}
\end{small}

Quelle: \url{https://de.wikipedia.org/wiki/BibTeX}
\end{frame}

\begin{frame}[fragile]{Bibliographie-Verzeichnis mit Bib\TeX}
Eine Ressource zitiert man im Text wie folgt:
\begin{lstlisting}[style=tex]
\cite{NAME}
\end{lstlisting}
Sie wird dann im Literaturverzeichnis aufgeführt.\pause

\bigskip
Sollen alle Einträge der Literaturdatenbank in das Literaturverzeichnis übernommen
werden, wenngleich sie nicht im Text zitiert wurden, kann dies wie folgt erreicht werden:
\begin{lstlisting}[style=tex]
\nocite{*}
\end{lstlisting}
\end{frame}