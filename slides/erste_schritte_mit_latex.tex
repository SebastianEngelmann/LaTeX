\section{Erste Schritte mit \LaTeX}

% ===== Aufbau eines LaTeX-Dokuments =====
\subsection{Aufbau eines \LaTeX{}-Dokuments}

\begin{frame}[fragile]{Aufbau eines \LaTeX{}-Dokuments}
	Die Minimalstruktur eines Quelltextes besteht aus zwei wesentlichen Teilen: Der \textbf{Präambel} und dem \textbf{Textkörper}. Alles was zwischen \texttt{\textbackslash documentclass} und \texttt{\textbackslash begin\{document\}} steht, zählt formal zur Präambel und alles danach zum Textkörper. 	
\begin{lstlisting}[style=tex]
\documentclass[Optionen]{Name}[Version] 
...
\begin{document}
...
\end{document}
\end{lstlisting}
Die Präambel enthält prinzipiell all das, was an allgemeinen Definitionen und zusätzlichen festlegungen für das gesamte Dokument vorgenommen wird.	
\end{frame}

\begin{frame}[fragile]{Ein pdf\LaTeX{}-Beispiel}
\begin{lstlisting}[style=tex]
%% Ein pdflatex-Beispiel
\documentclass[a5paper, ngerman, 11pt]{article}
\usepackage[T1]{fontenc}
\usepackage{lmodern}
\usepackage[utf8]{inputenc}
\usepackage{microtype}
\usepackage{geometry}
\usepackage{babel}
\usepackage{eurosym}
\usepackage{blindtext}
\title{Textsatz mit \LaTeX}
\author{Johannes Gutenberg}
\date{\today}
\begin{document}
\maketitle
\section{Einführung}
Erste Versuche mit dem Setzen eines \LaTeX-Dokuments für 0,--\,\euro.
\blindtext[2]
\end{document}
\end{lstlisting}
\end{frame}

\begin{frame}[fragile]{Erläuterung des pdf\LaTeX{}-Beispiels}
\begin{lstlisting}[style=tex, basicstyle=\tiny]
\documentclass[a5paper, ngerman, 11pt]{article}
   Dokumentenklasse article mit den globalen Optionen a5paper für das Papierformat, 
   ngerman für die neue deutsche Rechtschreibung und 11pt für die Grundschriftgröße.
\usepackage[T1]{fontenc}
   Lade das Paket fontenc für die Schriftkodierung (fontencoding) mit der Option T1, damit 
   zum einen Vektorschriften eingebunden werden und zum anderen eine Trennung an 
   deutschen Umlauten ermöglicht wird.
\usepackage[utf8]{inputenc}
   Lade das Paket inputenc für die Eingabekodierung (inputencoding) mit der Option utf8, 
   damit Umlaute und das ß direkt über die Tastatur eingegeben werden können.
\usepackage{babel}
   Lade Sprachpaket babel, welches die globale Option ngerman der Dokumentenklasse 
   auswertet.
\title{Textsatz mit \LaTeX}
   Definiere den Titel.
\author{Johannes Gutenberg}
   Definiere den Autor
\date{\today}
   Definiere das Datum. Es wird hier das aktuelle Datum verwendet.
\begin{document}
   Beginn des Textkörpers.
\maketitle
   Formatierten Titel ausgeben.
\section{Einführung}
   Abschnittsüberschrift.
\end{document}
   Ende des Dokuments.
\end{lstlisting}
\end{frame}

\begin{frame}[fragile]{Ein \XeLaTeX{}-Beispiel}
\begin{lstlisting}[style=tex]
%% Ein xelatex-Beispiel
\documentclass[a5paper, ngerman, 11pt]{article}
\usepackage{fontspec}
\fontspec[Mapping=tex-text]{Linux Libertine O}
\usepackage{microtype}
\usepackage{polyglossia}
\setmainlanguage[spelling=new]{german}
\usepackage{geometry}
\usepackage{dtk-logos}
\usepackage{blindtext}
\title{Textsatz mit \XeLaTeX}
\author{Johannes Gutenberg}
\date{\today}
\begin{document}
\maketitle
\section{Einführung}
Erste Versuche mit dem Setzen eines \XeLaTeX-Dokuments für 0,--.
\blindtext[2]
\end{document}
\end{lstlisting}
\end{frame}

\begin{frame}[fragile]{Ein \LuaLaTeX{}-Beispiel}
\begin{lstlisting}[style=tex]
%% Ein lualatex-Beispiel
\documentclass[a5paper, ngerman, 11pt]{article}
\usepackage{fontspec}
\setmainfont[Ligatures=TeX]{Linux Libertine O}
\usepackage{microtype}
\usepackage{babel}
\usepackage{geometry}
\usepackage{dtk-logos}
\usepackage{blindtext}
\title{Textsatz mit \LuaLaTeX}
\author{Johannes Gutenberg}
\date{\today}
\begin{document}
\maketitle
\section{Einführung}
Erste Versuche mit dem Setzen eines \LuaLaTeX-Dokuments für 0,--.
\blindtext[2]
\end{document}
\end{lstlisting}
\end{frame}