\section{Quellcode einbinden}

% ////////////////// Verbatim //////////////
\subsection{Verbatim}
\begin{frame}[fragile]{Verbatim}
Wenn Text nicht vom \LaTeX-Compiler interpretiert werden soll, kann man ihn in ein \textbf{verb}-Befehl setzen:

\begin{lstlisting}[style=tex]
\verb| nicht interpretierter Text |
\end{lstlisting}
Dabei muss dem Befehl mitgeteilt werden, wann dies endet. Dies geschieht durch ein Zeichen, das den Bereich einleitet und abschlie{\ss}t.

%\pause
\vspace{2ex}
Für mehrere Zeilen gibt es die \textbf{verbatim}-Umgebung:

\begin{lstlisting}[style=tex]
\usepackage{verbatim}

\begin{document}
  \begin{verbatim}
  diese Zeilen
  sollen nicht interpretiert
  werden       _ & \ -- % Kommentar
  \end{verbatim}
\end{document}
\end{lstlisting}
\end{frame}

% ////////////////// Quellcode inline einbinden //////////////
\subsection{Lstlistings}
\begin{frame}[fragile]{Lstlistings: Quellcode inline einbinden}
%Um Programmcode in das Dokument einzubinden, möchte man sprachspezifisches \textbf{Syntax-Highlighting} und Zeilennummerierungen haben.
Um Programmcode in das Dokument einzubinden, kann man das Paket \textbf{\texttt{listings}} verwenden.
%\pause
\vspace{2ex}
%Dafür gibt es das \textbf{listings}-Paket:

\begin{lstlisting}[style=tex]
\usepackage{listings}

\begin{document}
  \begin{lstlisting}[language=python, caption=Beschriftung]
      % hier steht Python-Code
  \end{lstlitsing}
\end{document}\end{lstlisting}

%Und so sähe Haskell-Code aus:
%\begin{lstlisting}[style=hs, caption= Fakultät in Haskell]
%fak :: [Integer] -> [Integer]
%fak 0 = 1
%fak n = n * fak(n-1)
%\end{lstlisting}

So sieht dann Python-Code aus:
\begin{lstlisting}[language=python, caption = Berechnung der n-ten Fibonacci in Python durch Rekursion]
def fib_rek(n):
    if n <= 0:
        raise Exception
    elif n == 1 or n == 2:
        return 1
    else:
        return fib_rek(n-1) + fib_rek(n-2)
\end{lstlisting}

% \begin{lstlisting}[style=py, caption=Beschriftung]
% def myFunction():
%     # hier kann man wild coden
%     return True\end{lstlisting}
\end{frame}

% ////////////////// Quellcode extern einbinden //////////////
\begin{frame}[fragile]{Quellcode extern einbinden}
In der Regel programmiert man in einer Datei und möchte Teile dieser im \LaTeX-Dokument einbinden. Dies geht mittels \textbf{lstinputstring}:

\begin{lstlisting}[style=tex, caption=lstlisting]
\lstinputlisting[language=latex, caption={Beschriftung}]{pfad/filename.tex}\end{lstlisting}

%\pause 
Man kann auch nur ein paar Zeilen des Programmcodes einbinden:

\begin{lstlisting}[style=tex, firstnumber=2]
\lstinputlisting[language=latex, linerange=23-42, firstnumber=23]{pfad/filename.tex}\end{lstlisting}

\begin{lstlisting}[style=tex, firstnumber=3]
\lstinputlisting[language=latex, firstline=23, lastline=42]{pfad/filename.tex}\end{lstlisting}

\begin{itemize}
\item \textbf{linerange} -- Bereich im Quellcode
\item \textbf{firstline} -- Beginn des einzubindenden Quellcodes
\item \textbf{lastline} -- Ende des einzubindenden Quellcodes 
\item \textbf{firstnumber} -- Beginn der Zeilen-Nummerierung
\end{itemize}
\end{frame}

% ////////////////// Liste der Programmcodes //////////////
\subsection{Liste der Programmcodes}
\begin{frame}[fragile]
Eine Liste der \textbf{lstlistings} wird wie folgt erstellt:

\begin{lstlisting}[style=tex]
\lstlistoflistings\end{lstlisting}
\end{frame}

