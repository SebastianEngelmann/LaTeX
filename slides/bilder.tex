\section{Bilder}
% ////////////////// Bilder einbinden //////////////
\subsection{Bilder einbinden}
\begin{frame}[fragile]{Bilder}
\begin{minipage}{0.6\linewidth}
Benötigt folgendes Paket:
\begin{lstlisting}[style=tex]
\usepackage{graphicx}
\end{lstlisting}

Ein Bild bindet man mit folgendem Code ein:

\begin{lstlisting}[style=tex]
\includegraphics[OPTION]{path/name}\end{lstlisting}

\pause
Options:
\begin{itemize}
\item \verb|scale=1| (Werte $>0$)
\item \verb|width=0.5\linewidth|
\item \verb|keepaspectratio=true|
\end{itemize}

\pause
Beispiel:
\begin{lstlisting}[style=tex]
\includegraphics[width=\linewidth]%
   {img/latex_is_beautiful}\end{lstlisting}
\end{minipage}
\hfill
\begin{minipage}{0.35\linewidth}
\pause
\includegraphics[width=\linewidth]{img/latex_is_beautiful}
\end{minipage}
\end{frame}

% ////////////////// Bildumgebung //////////////
\subsection{Bildumgebung}
\begin{frame}[fragile]{Bildumgebung}   
\begin{multicols}{2}
\begin{itemize}
\item erlaubt eine Bildbeschreibung
\item ermöglicht einfache Zentrierung
\item wird automatisch in das Bildverzeichnis übernommen
\item Bildquelle: \url{https://eu.fotolia.com/Content/Comp/82385255}
\end{itemize}

\begin{figure}
\centering
\includegraphics[width=0.45\linewidth]{img/latex_is_beautiful}
\caption[awesome_image]{Bildbeschreibung}
\label{fig:awesome_image}
\end{figure}
\end{multicols}
\vspace{-2ex}
\begin{lstlisting}[style=tex]
\begin{figure}
   \centering
   \includegraphics[width=0.5\linewidth]{img/latex_is_beautiful}
   \caption[awesome_image]{Bildbeschreibung}
   \label{fig:awesome_image}
\end{figure}\end{lstlisting}
\end{frame}

% ////////////////// Bildverzeichnis //////////////
\subsection{Bildverzeichnis}
\begin{frame}[fragile]{Bildverzeichnis}
Ein Bildverzeichnis erstellt \LaTeX{} automatisch mittels:

\begin{lstlisting}[style=tex]
\listoffigures
\end{lstlisting}

%\listoffigures
\end{frame}