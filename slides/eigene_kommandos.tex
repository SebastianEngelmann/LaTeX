\section{Eigene Kommandos und Umgebungen}

% ////////////////// Eigene Kommandos erstellen //////////////
\subsection{Eigene Kommandos erstellen}

\begin{frame}[fragile]{Eigene Kommandos erstellen}
Man kann sich eigene Kommandos erstellen, um z.\,B. Schreibaufwand einzusparen.

\begin{lstlisting}[style=tex]
\newcommand{\name}{Was es tun soll}\end{lstlisting}

\pause Hier ein Beispiel:
\begin{lstlisting}[style=tex]
\newcommand\zz{\ensuremath{\raisebox{+0.25ex}{Z}% zu-zeigen-Symbol
   \kern-0.4em\raisebox{-0.25ex}{Z}%
   \;\xspace}%
}\end{lstlisting}

\vspace{1ex}\pause Und so sieht es aus: $\zz$
\end{frame}

\begin{frame}[fragile]{Eigene Kommandos}
Es lassen sich auch Kommandos erstellen, welche Argumente erhalten und diese benutzen:

\begin{lstlisting}[style=tex]
\newcommand{\name}[Argumentzahl]{was es mit dem Argument #1 tun soll}\end{lstlisting}

\pause Hier ein Beispiel:

\begin{lstlisting}[style=tex]
\newcommand{\Quellcode}[3]{\lstinputlisting[%
   language=#2,   % 2. Argument: filename
   caption={#3}] % 3. Argument: Beschriftung
   {#1.#2}}         % 1. Argument: path\end{lstlisting}

\pause In der Anwendung:
\begin{lstlisting}[style=tex]
\Quellcode{src/myFile}{py}{Ein Python-Programm}\end{lstlisting}
\end{frame}

% ////////////////// Eigene Umgebungen erstellen //////////////
\subsection{Eigene Umgebungen erstellen}
\begin{frame}[fragile]{Eigene Umgebungen erstellen}
Möchte man eine bereits vorhandene Umgebung modifizieren, geht dies nicht mittels Kommando, sondern mit \verb|\newenvironment| .

\begin{lstlisting}[style=tex]
\newenvironment{name}[Argumentzahl]{Befehlsbeginn}{Befehlsende}\end{lstlisting}

\pause Nachfolgend ein Beispiel:

\begin{lstlisting}[style=tex]
\newenvironment{Magic}[1][Pink]% hat 1 optionales Arg., Standardwert: Pink
   {\begin{center}\begingroup\textcolor{#1}}%
   {\endgroup\end{center}}\end{lstlisting}
\pause

\begin{multicols}{2}
Hier ein Beispiel:
\begin{lstlisting}[style=tex]
\begin{Magic}
   mein toller Text
\end{Magic}
\end{lstlisting}

Und so sieht es aus:
\vspace{0.75ex}
\begin{Magic}
   mein toller Text
\end{Magic}
\vspace{0.75ex}
\end{multicols}
\end{frame}