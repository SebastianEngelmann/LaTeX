\section{Querverweise \& Hyperlinks}
\begin{frame}[fragile]{Querverweise \& Hyperlinks}
Um Querverweise im Dokument zu setzen, muss man folgende Pakete einbinden:
\begin{lstlisting}[style=tex]
\usepackage[        % Verlinkungen
  colorlinks,            % farbige Schrift, statt farbiger Rahmen
  linktocpage,        % verlinkt im Abb.Verzeichnis Seitenzahl statt Bildunterschrift
  linkcolor=blue       % setzt Farbe der Links auf blau
]{hyperref}
\usepackage{url}  % für Webadressen: "\url{http://www.example.com}"
\end{lstlisting}\pause

\bigskip
Einen dokumentinternen Verweis\label{LABELNAME} kann man wie folgt erstellen:
\begin{lstlisting}[style=tex]
\label{LABELNAME} % setzt einen Verweis an die entsprechende Stelle.
\end{lstlisting}

Auf diesen kann in einer anderen Stelle referenziert werden:
\begin{lstlisting}[style=tex]
Hier steht ein Verweis zur Seite \ref{LABELNAME}.
\end{lstlisting}\pause

\bigskip
$\Rightarrow$ Hier steht ein Verweis zur Seite \ref{LABELNAME}.
\end{frame}

\begin{frame}[fragile]{Querverweise \& Hyperlinks}
Auch \textbf{URL}s kann man einfach hinzufügen:
\begin{lstlisting}[style=tex]
\url{http://latex.org}
\end{lstlisting}
So sieht es aus: \url{http://latex.org}\pause

\bigskip
Ein \textbf{Verzeichnis der Querverweise} wird automatisch mit folgendem Befehl erstellt:
\begin{lstlisting}[style=tex]
\linktocpage
\end{lstlisting}
\end{frame}
