\section{Was ist \LaTeX?}

\begin{frame}{Was ist \LaTeX?}
\begin{itemize}
    \item \LaTeX{} ist eine Sammlung von \textbf{Makros}, die die Nutzung von \TeX{} ermöglicht
    \newline
    \item \TeX{} wiederum ist ein \textbf{Drucksatzsystem}, mit dem sich Texte formatieren lassen
    \newline
    \item Text und Formatierungselemente werden \textbf{direkt in den Quelltext} geschrieben
    \newline
    \item \textbf{Vorteile:} Einfachheit in der Darstellung komplexer Strukturen, mathematischer Formeln, Grafiken und ähnlichem
\end{itemize}
    
\end{frame}

\begin{frame}{Was ist \LaTeX?: Installation}
\begin{itemize}
\item \textbf{Windows:}
\begin{enumerate}
\item \textbf{MikTex Installer} installieren: \url{http://miktex.org/download}
\item einzelne \textbf{Pakete} installieren: \textit{"MiKTeX" > "Maintenance (Admin)" > "Package Manager (Admin)"}
\item \LaTeX{} \textbf{Editor} einrichten, z.B. TeXstudio oder Texmaker
\end{enumerate}
\item \textbf{Macintosh:}
\begin{enumerate}
\item \textbf{MacTex Installer} installieren: \url{http://www.tug.org/mactex/}
\item einzelne \textbf{Pakete} installieren: \textit{"TeX Live Utiility"}
\item \LaTeX{} \textbf{Editor} einrichten, z.B. TeXstudio oder Texmaker
\end{enumerate}
\end{itemize}
\end{frame}

\begin{frame}{Was ist \LaTeX?: Installation}
\begin{itemize}
\item \textbf{Linux:}
\begin{enumerate}
\item \$ sudo apt-get install texlive-full
\item \$ sudo apt-get install texlive-xetex
\item \$ sudo apt-get install texlive texlive-doc-de texlive-latex-extra texlive-lang-german
\item \$ sudo apt-get install latex-xcolor pgf tex-common texlive texlive-base texlive-base-bin texlive-common texlive-fonts-extra texlive-fonts-recommended texlive-lang-german texlive-latex-base texlive-latex-extra texlive-latex-recommended
\item \$ sudo apt-get install texlive-pictures dot2tex sketch libqtexengine1 texlive-humanities texlive-pstricks
\item \$ wget \url{http://mirror.ctan.org/graphics/pgf/contrib/tikz-qtree.zip}
\begin{itemize}
\item \textbf{im Ordner entpacken}
\end{itemize}

\item \$ sudo mv tikz-qtree/ /usr/share/texmf/tex/latex/
\item \$ sudo texhash
\end{enumerate}
\end{itemize}
\end{frame}
