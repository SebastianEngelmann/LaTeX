\section{Textstrukturierung}

% ===== Gliederung =====
\subsection{Gliederung \& Inhaltsverzeichnis}
\begin{frame}[fragile]{Gliederung \& Inhaltsverzeichnis}
%\begin{varwidth}{0.45\linewidth}
%Eine Überschrift wird mit folgendem Befehl erstellt:
Mit dem Makro \texttt{\textbackslash section} wird eine Überschrift in der ersten Gliederungsebene erzeugt:
\begin{lstlisting}[style=tex]
\section{head}
\end{lstlisting}
%\pause
Für eine Überschrift in der zweiten Gliederungsebene benutzen wir das Makro \texttt{\textbackslash subsection}:
\begin{lstlisting}[style=tex]
\subsection{subhead}
\end{lstlisting}
%\pause
In der dritten Gliederungsebene benutzen wir das Makro \texttt{\textbackslash subsubsection} für eine Überschrift:
\begin{lstlisting}[style=tex]
\subsubsection{subsubhead}
\end{lstlisting}
\end{frame}

% ===== Inhaltsverzeichnis =====
\begin{frame}[fragile]{Gliederung \& Inhaltsverzeichnis}
Das Inhaltsverzeichnis wird automatisch mit dem Makro
\begin{lstlisting}[style=tex]
\tableofcontents
\end{lstlisting}
erstellt. Dabei werden alle Überschriften übernommen.
%\pause
\newline
\newline
Möchte man nicht, dass eine Überschrift im Verzeichnis auftaucht, benutzt man das Sternchen:
\begin{lstlisting}[style=tex]
\section*{diese Überschrift taucht nicht auf}
\end{lstlisting}
\end{frame}

% ===== Absätze, Zeilenumbrüche =====
\subsection{Absätze \& Zeilenumbrüche}
\begin{frame}[fragile]{Absätze \& Zeilenumbrüche}

Ein Zeilenumbruch kann entweder durch eine \textbf{Leerzeile} im Quelltext oder durch das Makro \textbf{\textbackslash\texttt{par}}  erreicht werden.

\begin{itemize}
	\item Gliederung des Inhalts in Gedankengänge
	\item 1. Zeile ist eingerückt
	\item Abstand zwischen zwei Absätzen
\end{itemize}

%\begin{multicols}{2}
%\begin{lstlisting}[style=tex]
%\par
%\end{lstlisting}
%\begin{lstlisting}[style=tex]
%\newline
%\end{lstlisting}
%\begin{lstlisting}[style=tex]
%\linebreak
%\end{lstlisting}
%\begin{lstlisting}[style=tex]
%\\
%\end{lstlisting}
%\end{multicols}%\pause

Möchte man nicht, dass die erste Zeile eines Absatzes eingerückt wird, kann man folgende Zeile in die Präambel des Dokuments vornehmen:
\begin{lstlisting}[style=tex]
\parindent 0pt
\end{lstlisting}

Einen manuellen Zeilenumbruch ohne Abstand kann man mit folgenden Makros erzeugen:
\begin{lstlisting}[style=tex]
\newline              % Zeilenumbruch im selben Absatz
\linebreak
\\
\end{lstlisting}

\end{frame}

% ===== Auflistung, Aufzählung =====
\subsection{Auflistung \& Aufzählung}
\begin{frame}[fragile]{Auflistung \& Aufzählung}
Für unnummerierte Auflistungen kann man die Umgebung \textbf{itemize} benutzen. Sie kann bis zu vier Ebenen tief geschachtelt werden.
\begin{lstlisting}[style=tex]
\begin{itemize}
\item Stichpunkt 1
\item Stichpunkt 2
...
\end{itemize}
\end{lstlisting}%\pause
Für nummerierte Auflistungen kann man die Umgebung \textbf{enumerate} verwenden. Sie kann ebenfalls bis zu vier Ebenen tief geschachtelt werden.
\begin{lstlisting}[style=tex]
\begin{enumerate}
\item Stichpunkt 1
\item Stichpunkt 2
...
\end{enumerate}
\end{lstlisting}
\end{frame}

\begin{frame}[fragile]{Auflistung \& Aufzählung Nummerierung}
Die Aufzählungszeichen lassen sich auch verändern.
Dazu bindet man das Paket \textbf{enumerate} ein:
\begin{lstlisting}[style=tex]
\usepackage{enumerate}

\begin{document}
   \begin{enumerate}[a)]
   \item erste Ebene
      \begin{enumerate}[(i)]
         \item zweite Ebene mit der Einstellung [(i)]
      \end{enumerate}
   \end{enumerate}
\end{document}
\end{lstlisting}%\pause

\bigskip
Und so sieht es aus:
\begin{enumerate}[a)]
\item erste Ebene
\begin{enumerate}[(i)]
\item zweite Ebene mit der Einstellung [(i)]
\end{enumerate}
\end{enumerate}
\end{frame}

% ===== Tabulars =====
\subsection{Tabulars}
\begin{frame}[fragile]{Tabulars}
Einfache Tabellen können mit der Umgebung \textbf{tabular} erstellt werden.
\begin{lstlisting}[style=tex]
\begin{tabular}{Spaltendefinitionen}
Tabelleninhalt
\end{tabular}
\end{lstlisting}

In das Feld \{Spaltendefinitionen\} wird für jede Spalte die Formatierung angegeben. \\

\begin{itemize}
	\item \textbf{l} -- linksbündige Spalte (ohne Zeilenumbruch!) 
	\item \textbf{c} -- zentrierte Spalte (ohne Zeilenumbruch!)
	\item \textbf{r} -- rechtsbündige Spalte (ohne Zeilenumbruch!)
	\item \textbf{p\{Länge\}} -- entspricht der Definition von \texttt{\textbackslash parbox[c]\{Länge\}}, welche grundsätzlich im Blocksatz und mit Zeilenumbrüchen gesetzt wird
	\item \textbf{|} -- senkrechte Linie, kann beliebig erweitert werden zu ||...
\end{itemize}
%\begin{itemize}
%\item \textbf{c} -- für zentrierten Text (einzeilig)
%\item \textbf{l} -- für linksbündigen Text (einzeilig)
%\item \textbf{r} -- für rechtsbündigen Text (einzeilig)
%\item \textbf{p\{SIZEcm\}} -- Spalte soll SIZE cm breit sein und der Zelleninhalt wird automatisch umgebrochen
%\item \textbf{|} -- zieht eine vertikale Trennlinie zwischen zwei Spalten
%\end{itemize}
\end{frame}

\begin{frame}[fragile]{Tabulars}
Der Tabelleninhalt wird wie folgt gesetzt:
\begin{itemize}
\item \textbf{\&} -- trennt die Spalten voneinander
\item \textbf{\textbackslash\textbackslash} -- trennt die Zeilen voneinander
\item \textbf{\textbackslash hline} -- zieht eine horizontale Trennlinie zwischen zwei Zeilen
\end{itemize}

Beispiel:
\begin{lstlisting}[style=tex]
\begin{tabular}{| l | c | r |}
Spalte 1 & Spalte 2 & Spalte 3 \\
\hline  % horizontale Trennlinie
1 & 2 & 3 \\
\end{tabular}
\end{lstlisting}

\bigskip
Und so sieht es aus:
%\begin{center}
\renewcommand{\arraystretch}{1.2}
\begin{tabular}{| l | c | r |}
Spalte 1 & Spalte 2 & Spalte 3 \\
\hline
1 & 2 & 3 \\
\end{tabular}
%\end{center}

\bigskip
Dabei richtet sich die Tabellenbreite nach dem Inhalt.
\end{frame}

\begin{frame}[fragile]{Tabulars}
Möchte man, dass die Tabelle eine feste Breite hat und Zelleninhalt automatisch umgebrochen wird, benutzt man \textbf{tabularx}
\begin{lstlisting}[style=tex]
\usepackage{tabularx}

\begin{document}
   \begin{tabularx}{\linewidth}{| X | X | p{2cm} |}
   \hline
   ... & ... & ... \\
   \hline
   \end{tabularx}
\end{document}
\end{lstlisting}%\pause

So sieht es aus:\vspace{3pt}
\begin{tabularx}{\linewidth}{| X | X | p{2cm} |}
\hline
Diese Spalte ist genauso breit wie die rechts neben ihr stehende. & Der Text wird in Blocksatz gesetzt.  & Diese Spalte hat eine feste Breite von 2\,cm. \\
\hline
\end{tabularx}

\bigskip
Die Gesamtbreite wird zu gleichen Teilen auf alle Spalten, die mit \textbf{X} deklariert wurden, verteilt.
\end{frame}

\begin{frame}[fragile]{Tabulars}
Zeilen und Spalten zu einer verschmelzen:

\begin{lstlisting}[style=tex]
\usepackage{multirow}   % für multirow in tabulars
\end{lstlisting}%\pause

Beispiel:
\begin{lstlisting}[style=tex]
\begin{tabular}{|c|c|c|}
  \hline
  \multicolumn{2}{|c|}{Kuchen} & Hefe \\
  \hline
  Zeit & \multirow{2}{*}{zwei Zeilen} & Zucker \\
  \cline{1-1}\cline{3-3}  % vertikale Trennlinie
  Sieb & & Mehl \\
  \hline
\end{tabular}
\end{lstlisting}

%\begin{center}
So sieht es aus:
\renewcommand{\arraystretch}{1.2}
\begin{tabular}{|c|c|c|}
\hline
\multicolumn{2}{|c|}{eine Spalte} & Hefe \\
\hline
Zeit & \multirow{2}{*}{eine Zeile} & Zucker \\
\cline{1-1}\cline{3-3}
Sieb & & Mehl \\
\hline
\end{tabular}
%\end{center}
\end{frame}

\begin{frame}[fragile]{Tabellenverzeichnis}
Mit \textbf{\texttt{\textbackslash listoftables}} kann ein Tabellenverzeichnis erstellt werden.
Die Tabellen müssen sich dafür in der \textbf{table}-Umgebung befinden.

\begin{lstlisting}[style=tex]
\begin{table}
   \caption{\LaTeX{} Kursübersicht}
   \label{tab01}
   \begin{tabular}{|c|c|c|}
      \hline
      Zeitpunkt & Kursleiter & Titel \\
      \hline 
      ...   
   \end{tabular}
\end{table}
\end{lstlisting}\label{tab01}%\pause

\begin{itemize}
\item Der Befehl \textbf{\textbackslash listoftables} erstellt automatisch das Tabellenverzeichnis.
\item Unsere Tabelle erzeugt den Eintrag ,,\textbf{\LaTeX{}~Kursübersicht} S.~\ref{tab01}``.
\end{itemize}
\end{frame}

% ////////////////// Spaltenlayout //////////////
\subsection{Spaltenlayout}
\begin{frame}[fragile]{Spaltenlayout}
\begin{itemize}
\item Einzelne Bereiche einer Seite können in mehrere Spalten aufgeteilt werden.
\item Dafür wird das \textbf{multicol}-Paket benötigt.
\end{itemize}

\begin{lstlisting}[style=tex]
\usepackage{multicol}

\begin{document}
   \begin{multicols}{ANZAHL}
   Inhalt
   \end{multicols}
   ...
\end{document}
\end{lstlisting}

{\itshape
\begin{multicols}{3}
Der Inhalt (z.B. Text und Bilder) werden auf die Anzahl der Spalten gleichmäßig verteilt. Satzart: linksbündig. Faustregel: Je schmaler das Layout, desto weniger Spalten sollte man nehmen. Der Spalten-Zwischenraum wird automatisch angepasst.
\end{multicols}}
\end{frame}

% ////////////////// Minipages //////////////
\subsection{Minipages}
\begin{frame}[fragile]{Minipages}
Minipages geben neben Spalten und Tabellen die Möglichkeit, Inhalte zusammengehörig in einer Art \textbf{Container} zu anderen Inhalten auszurichten und ihnen eine feste Breite zu geben.
\begin{lstlisting}[style=tex]
\begin{minipage}[ÄUSSERE POSITION][HÖHE][INNERE POSITION]{BREITE}
Inhalt
\end{minipage}\end{lstlisting}
\begin{itemize}
\item ÄUSSERE POSITION richtet die Minipage relativ zur aktuellen Grundlinie aus:
\begin{itemize}
\item c = center
\item t = top
\item b = bottom \newline
\end{itemize}
\item HÖHE, ist eine gültige Längenangabe, durch die die Gesamthöhe der Minipage bestimmt wird. \newline
\item INNERE POSITION, richtet den Inhalt der Minipage innerhalb der angegebenen HÖHE aus.
\end{itemize}
\end{frame}

\begin{frame}[fragile]{Minipages}
Beispiel:
\begin{lstlisting}[style=tex]
\begin{minipage}{0.6\linewidth}
   Inhalt 1
\end{minipage}
\hfill % schiebt die nachfolgende Minipage an den rechten Layoutrand
\begin{minipage}{0.3\linewidth}
   Inhalt 2
\end{minipage}
\end{lstlisting}

\begin{minipage}{0.7\linewidth}
\itshape Die Minipages richten sich standardmäßig vertikal mittig zueinander aus. Der Zwischenraum beträgt im Beispielcode $10\%$ der Layoutbreite.
\end{minipage}
\hfill 
\begin{minipage}{0.25\linewidth}
   \includegraphics[width=\linewidth]{img/latex_is_beautiful}
\end{minipage}
\end{frame}