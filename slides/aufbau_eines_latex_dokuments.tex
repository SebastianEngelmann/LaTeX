\section{Aufbau eines \LaTeX{}-Dokuments}

% ===== Struktur =====
\subsection{Struktur}

\begin{frame}[fragile]{Struktur}
	Die Minimalstruktur eines Quelltextes besteht aus zwei wesentlichen Teilen: Der \textbf{Präambel} und dem \textbf{Textkörper}. Alles was zwischen \texttt{\textbackslash documentclass} und \texttt{\textbackslash begin\{document\}} steht, zählt formal zur Präambel und alles danach zum Textkörper. 	
\begin{lstlisting}[style=tex]
\documentclass[Optionen]{Name}[Version] 
...
\begin{document}
...
\end{document}
\end{lstlisting}
Die Präambel enthält prinzipiell all das, was an allgemeinen Definitionen und zusätzlichen festlegungen für das gesamte Dokument vorgenommen wird.	
\end{frame}

% ===== Beispiele =====
\subsection{Beispiele}

\begin{frame}[fragile]{Ein pdf\LaTeX{}-Beispiel}
\begin{lstlisting}[style=tex]
%% Ein pdflatex-Beispiel
\documentclass[a5paper, ngerman, 11pt]{article}
\usepackage[T1]{fontenc}
\usepackage{lmodern}
\usepackage[utf8]{inputenc}
\usepackage{microtype}
\usepackage{geometry}
\usepackage{babel}
\usepackage{eurosym}
\usepackage{blindtext}
\title{Textsatz mit \LaTeX}
\author{Johannes Gutenberg}
\date{\today}
\begin{document}
\maketitle
\section{Einführung}
Erste Versuche mit dem Setzen eines \LaTeX-Dokuments für 0,--\,\euro.
\blindtext[2]
\end{document}
\end{lstlisting}
\end{frame}

\begin{frame}[fragile]{Erläuterung des pdf\LaTeX{}-Beispiels}
\begin{lstlisting}[style=tex, basicstyle=\tiny]
\documentclass[a5paper, ngerman, 11pt]{article}
   Dokumentenklasse article mit den globalen Optionen a5paper für das Papierformat, 
   ngerman für die neue deutsche Rechtschreibung und 11pt für die Grundschriftgröße.
\usepackage[T1]{fontenc}
   Lade das Paket fontenc für die Schriftkodierung (fontencoding) mit der Option T1, damit 
   zum einen Vektorschriften eingebunden werden und zum anderen eine Trennung an 
   deutschen Umlauten ermöglicht wird.
\usepackage[utf8]{inputenc}
   Lade das Paket inputenc für die Eingabekodierung (inputencoding) mit der Option utf8, 
   damit Umlaute und das ß direkt über die Tastatur eingegeben werden können.
\usepackage{babel}
   Lade Sprachpaket babel, welches die globale Option ngerman der Dokumentenklasse 
   auswertet.
\title{Textsatz mit \LaTeX}
   Definiere den Titel.
\author{Johannes Gutenberg}
   Definiere den Autor
\date{\today}
   Definiere das Datum. Es wird hier das aktuelle Datum verwendet.
\begin{document}
   Beginn des Textkörpers.
\maketitle
   Formatierten Titel ausgeben.
\section{Einführung}
   Abschnittsüberschrift.
\end{document}
   Ende des Dokuments.
\end{lstlisting}
\end{frame}

\begin{frame}[fragile]{Ein \XeLaTeX{}-Beispiel}
\begin{lstlisting}[style=tex]
%% Ein xelatex-Beispiel
\documentclass[a5paper, ngerman, 11pt]{article}
\usepackage{fontspec}
\fontspec[Mapping=tex-text]{Linux Libertine O}
\usepackage{microtype}
\usepackage{polyglossia}
\setmainlanguage[spelling=new]{german}
\usepackage{geometry}
\usepackage{dtk-logos}
\usepackage{blindtext}
\title{Textsatz mit \XeLaTeX}
\author{Johannes Gutenberg}
\date{\today}
\begin{document}
\maketitle
\section{Einführung}
Erste Versuche mit dem Setzen eines \XeLaTeX-Dokuments für 0,--.
\blindtext[2]
\end{document}
\end{lstlisting}
\end{frame}

\begin{frame}[fragile]{Ein \LuaLaTeX{}-Beispiel}
\begin{lstlisting}[style=tex]
%% Ein lualatex-Beispiel
\documentclass[a5paper, ngerman, 11pt]{article}
\usepackage{fontspec}
\setmainfont[Ligatures=TeX]{Linux Libertine O}
\usepackage{microtype}
\usepackage{babel}
\usepackage{geometry}
\usepackage{dtk-logos}
\usepackage{blindtext}
\title{Textsatz mit \LuaLaTeX}
\author{Johannes Gutenberg}
\date{\today}
\begin{document}
\maketitle
\section{Einführung}
Erste Versuche mit dem Setzen eines \LuaLaTeX-Dokuments für 0,--.
\blindtext[2]
\end{document}
\end{lstlisting}
\end{frame}

% ===== Dokumentenklassen =====
\subsection{Dokumententypen}

\begin{frame}[fragile]{Dokumentenklassen}
In jedem \LaTeX{}-Dokument muss die \textbf{Dokumentenklasse} angegeben werden (erster Befehl des Dokuments). Jede Dokumentenklasse hat \textbf{verschiedene Eigenschaften}. Die Dokumentenklasse (in geschweiften Klammern) wird über den Befehl \texttt{\textbackslash documentclass} ausgewählt. Der ausgewählten Klasse können Optionen (in eckigen Klammern) mitgegeben werden, die von der Dokumentenklasse selbst als auch von den verwendeten Paketen ausgewertet werden können.
%\begin{varwidth}{0.45\linewidth}
\begin{lstlisting}[style=tex]
\documentclass[Optionen]{Dokumentenklasse}
\end{lstlisting}
\end{frame}

% ////////////////// Typen Übersicht //////////////
\begin{frame}{Standardklassen}
\begin{itemize}
	\item \textbf{book:} Drucklayout standardmä{\ss}ig zweiseitig; z.B.: automatische Kopfzeile mit Seitenzahl
	\begin{itemize}
		\item Titel auf eigener Seite
		\item Seitenaufzählungen mit römischen und arabischen Ziffern
		\item Ebenen \textbackslash\texttt{part}, \textbackslash\texttt{chapter}, \textbackslash\texttt{section}, \textbackslash\texttt{subsection}, \textbackslash\texttt{subsubsection} und \textbackslash\texttt{paragraph}
	\end{itemize}
\item \textbf{article:} geeignet für kurze technische Artikel
	\begin{itemize}
		\item Titel auf eigener Seite
		\item Seitenzählung mit arabischen Ziffern
		\item Ebenen \textbackslash\texttt{part}, \textbackslash\texttt{section}, \textbackslash\texttt{subsection}, \textbackslash\texttt{paragraph}
		\item Fortlaufende Nummerierungen der Abbildungen, Tabellen, Fu{\ss}noten und Gleichungen
	\end{itemize}
\item \textbf{report:} geeignet für längere technische Artikel
	\begin{itemize}
		\item wie \texttt{article}, jedoch mit \textbackslash\texttt{chapter}
	\end{itemize}
\item \textbf{letter:} geeignet zum Schreiben von Briefen
	\begin{itemize}
		\item Keine Ebenen
		\item Elemente, die ein Brief aufweist, wie Absender, Anschrift, usw.
	\end{itemize}
\end{itemize}
\end{frame}

\begin{frame}{Koma-Script}
	Die Koma-Script-Reihe ist eine Sammlung von Klassen und Paketen, die besonders die typografischen Gepflogenheiten eines europäischen Layouts berücksichtigen (\url{https://komascript.de}). Sie besteht aus folgenden Dokumentenklassen: \\
	%\vspace{0.5cm}
	\begin{itemize}
		\item \textbf{scrbook}
		\item \textbf{scrartcl}
		\item \textbf{scrreprt}
		\item \textbf{scrlettr2}
	\end{itemize}

\end{frame}

% ///////////// Pakete ///////////////
\subsection{Pakete}
\begin{frame}[fragile]{Pakete}
\TeX{} und \LaTeX{} besitzen einen Grundstock an Befehlen. Wenn man darüber hinaus Modifizierungen vornehmen möchte, muss man zusätzliche Pakete einbinden, die weitere Befehle zur Verfügung stellen.\pause

\begin{lstlisting}[style=tex]
\usepackage{fontspec}
% Schriftpaket, funktioniert nur mit den neueren Compilern z.B. XeLaTeX
\usepackage{microtype}
% verbessert die Worttrennung
\usepackage[ngerman]{babel}
% Spracheinstellung: richtige Silbentrennung.
\usepackage{lmodern}
% verändert verwendete Schriftart, damit sie weniger pixelig ist
\end{lstlisting}
\end{frame}

% ////////////// Begin Dokument //////////////
\subsection{Das Dokument}
\begin{frame}[fragile]{Das Dokument}
Im Kopf des Dokuments kann man Einstellungen vornehmen und zusätzliche Pakete einbinden. Der eigentliche Inhalt folgt dann in der \textbf{document}-Umgebung:
\begin{lstlisting}[style=tex]
\documentclass[paper=a4, 10pt, ngerman]{scrartcl}

\begin{document}
Mein Inhalt
\end{document}\end{lstlisting}
\end{frame}

% //////// Titlepage //////////////
\subsection{Titelseite}
\begin{frame}[fragile]{Titelseite}
Die für ein Dokument wichtige Angaben (wie z.B. Titel, Autor und Datum) lassen sich sehr einfach erstellen und werden dann automatisch vorformatiert.
\begin{lstlisting}[style=tex]
\begin{titlepage}
\title{Crashkurs LaTeX}
\subtitle{Mentoring WiSe 2016}
\subject{Computer Science}
\author{Anja Wolffgramm, Diane Hanke}
\date{25. November 2016}
\end{titlepage}

\maketitle % erstellt die Titelseite
\end{lstlisting}

% \textbf{Anwendung:}
% \vspace{-2ex}
% \begin{lstlisting}[style=tex]
% \begin{titlepage}
%   ...
% \end{titlepage}

% \begin{document}
%    \maketitle
% \end{document}
% \end{lstlisting}
% Wendet man die Zeilen 1 -- 7 vor Beginn des Dokuments an, wird eine extra Cover-Seite erstellt.

% \begin{lstlisting}[style=tex]
% \begin{document}
%   \begin{titlepage}
%     ...
%   \end{titlepage}

%   \maketitle
% \end{document}
% \end{lstlisting}
% Wendet man es innerhalb des Dokuments an, wird nachfolgender Inhalt auf die selbe Seite gesetzt.
\end{frame}

% ===== Sonderzeichen =====
\subsection{Sonderzeichen}

\begin{frame}{Sonderzeichen in \LaTeX}
Nicht alle Zeichen der Tastatur stehen für die Eingabe zur Verfügung, da einige Sonderzeichen als \TeX{}-Steuerzeichen verwendet werden. Die nachfolgende Tabelle fasst die reservierten Zeichen zusammen.
\begin{table}
\caption{Eingabe von Sonderzeichen in \LaTeX}
	\begin{tabular}{cll}
		\hline
		Zeichen & Eingabe & Bedeutung \\
		\hline
		\{ & \textbackslash\{ & Beginn einer Gruppe \\
		\} & \textbackslash\} & Ende einer Gruppe \\
		\# & \textbackslash\# & Parameter \\
		\& & \textbackslash\& & Trenner bei Matrizen und Tabellen \\
		\_ & \textbackslash\_ & Indizes im mathematischen Modus \\
		\% & \textbackslash\% & Leitet Kommentare ein \\
		\$ & \textbackslash\$ & Mathematischer Zeilenmodus \\
		\textbackslash & \textbackslash\texttt{textbackslash} & Makrobeginn \\
		\textasciitilde & \textbackslash\texttt{textasciitilde} & Leerzeichen ohne Umbruch \\
	    \textasciicircum & \textbackslash\texttt{textasciicircum} & Exponent im mathematischen Modus \\
		\hline
	\end{tabular}
\end{table}
\end{frame}